\documentclass[a4paper, 11pt, titlepage, english]{article}

\usepackage{babel, textcomp}
\usepackage[latin1]{inputenc}
\usepackage[T1]{fontenc, url}
\usepackage{amsmath, amssymb}
\usepackage{amsbsy, amsfonts}
\usepackage{graphicx, color, xcolor}
\usepackage{framed, parskip}
\usepackage{flafter, caption, multicol}
\usepackage{verbatim, listings, fancyvrb}

%\DeclareCaptionLabelSeparator{colon}{. }
\renewcommand{\captionfont}{\small\sffamily}
\renewcommand{\captionlabelfont}{\bf\sffamily}
\setlength{\captionmargin}{20pt}

\setcounter{tocdepth}{2}

\DeclareMathAlphabet{\mathbfit}{OML}{cmm}{b}{it}

\definecolor{javared}{rgb}{0.6,0,0} % for strings
\definecolor{javagreen}{rgb}{0.25,0.5,0.35} % comments
\definecolor{javapurple}{rgb}{0.5,0,0.35} % keywords
\definecolor{javadocblue}{rgb}{0.25,0.35,0.75} % javadoc

\lstset{language=python,
basicstyle=\ttfamily\scriptsize,
keywordstyle=\color{javapurple},%\bfseries,
stringstyle=\color{javared},
commentstyle=\color{javagreen},
morecomment=[s][\color{javadocblue}]{/**}{*/},
morekeywords={super, with},
% numbers=left,
% numberstyle=\tiny\color{black},
stepnumber=2,
numbersep=10pt,
tabsize=4,
showspaces=false,
captionpos=b,
showstringspaces=false,
frame= single,
breaklines=true}
% 
% \makeatletter
% \def\lst@lettertrue{\let\lst@ifletter\iffalse}
% \makeatother

\usepackage{geometry}
\geometry{headheight=0.01mm}
\geometry{top=20mm, bottom=20mm, left=34mm, right=34mm}



\renewcommand{\arraystretch}{2}
\setlength{\tabcolsep}{10pt}
\makeatletter
\renewcommand*\env@matrix[1][*\c@MaxMatrixCols c]{%
  \hskip -\arraycolsep
  \let\@ifnextchar\new@ifnextchar
  \array{#1}}
\makeatother

\newcommand{\bt}[1]{\mathbfit{#1}}
\newcommand{\p}{\partial}
\renewcommand{\d}{\text{d}}
\newcommand{\D}{\text{D}}
\newcommand{\N}{\text{N}}
\newcommand{\R}{\text{R}}
  \renewcommand{\c}{\cdot}

\usepackage{calc}

\title{INF3331 - Week 8}
\author{Jonas van den Brink \\ \texttt{j.v.d.brink@fys.uio.no}}

\begin{document}
\pagestyle{empty}
\subsection*{Jonas van den Brink, jvbrink}
\texttt{j.v.d.brink@fys.uio.no}
\subsection*{8.1---Assignment and in-place modifications of NumPy arrays}
The code does not work as intended because the assignments \verb+y = x+ and \verb+z = x+ produces \emph{shallow copies} of the NumPy-array \verb+x+. This effectively means that the variables \verb+y+ and \verb+z+ point to the same array in memory as \verb+x+, so changing one of them, will change all of them simultaneously.

To fix this, we should instead use the \verb+npdarray+-method \verb+copy()+, which returns a deep copy of the array. So we have the code
\begin{lstlisting}
from scitools.numpytools import *
x = sequence(0, 1, 0.5)
y = x.copy(); y *=2; y += 1
z = x.copy(); z *= 4; 
print x, y, z
\end{lstlisting}
This is equivalent to the syntax
\begin{lstlisting}
from scitools.numpytools import *
x = sequence(0, 1, 0.5)
y = 2*x + 1
z = 4*x - 4
print x, y, z
\end{lstlisting}
Both of these programs produce the output
\begin{lstlisting}
[ 0.   0.5  1. ] [ 1.  2.  3.] [-4. -2.  0.]
\end{lstlisting}

\subsection*{8.2---Process comma-separated numbers in a file}
The exercise asked for a compact script, so readability has not been emphasized
\lstinputlisting{process_spreadsheet.py}

\clearpage

\subsection*{8.3---Matrix-vector multiply with NumPy arrays}
\lstinputlisting{matvec.py}

\subsection*{8.4---Replace lists by NumPy arrays}
Simply changing lists to NumPy-arrays makes little difference. I instead use the function \verb+np.genfromtxt+ to generate a NumPy-array that behaves much like a dictionary.

\lstinputlisting{convert2_wNumPy.py}

\clearpage

\subsection*{8.5---Rock, Paper, Scissors}

\lstinputlisting{roshambo.py}



\end{document}