\documentclass[a4paper, 11pt, notitlepage, english]{article}

\usepackage{babel}
\usepackage[utf8]{inputenc}
\usepackage[T1]{fontenc, url}
\usepackage{textcomp}
\usepackage{amsmath, amssymb}
\usepackage{amsbsy, amsfonts}
\usepackage{graphicx, color}
\usepackage{parskip}
\usepackage{framed}
\usepackage{amsmath}
\usepackage{xcolor}
\usepackage{multicol}
\usepackage{url}
\usepackage{flafter}


\usepackage{geometry}
\geometry{headheight=0.01mm}
\geometry{top=24mm, bottom=29mm, left=39mm, right=39mm}

\renewcommand{\arraystretch}{2}
\setlength{\tabcolsep}{10pt}
\makeatletter
\renewcommand*\env@matrix[1][*\c@MaxMatrixCols c]{%
  \hskip -\arraycolsep
  \let\@ifnextchar\new@ifnextchar
  \array{#1}}
%
% Parametere for inkludering av kode fra fil
%
\usepackage{listings}
\lstset{language=python}
\lstset{basicstyle=\ttfamily\small}
\lstset{frame=single}
\lstset{keywordstyle=\color{red}\bfseries}
\lstset{commentstyle=\itshape\color{blue}}
\lstset{showspaces=false}
\lstset{showstringspaces=false}
\lstset{showtabs=false}
\lstset{breaklines}

%
% Definering av egne kommandoer og miljøer
%
\newcommand{\dd}[1]{\ \text{d}#1}
\newcommand{\f}[2]{\frac{#1}{#2}} 
\newcommand{\beq}{\begin{equation}}
\newcommand{\eeq}{\end{equation}}
\newcommand{\bra}[1]{\langle #1|}
\newcommand{\ket}[1]{|#1 \rangle}
\newcommand{\braket}[2]{\langle #1 | #2 \rangle}
\newcommand{\braup}[1]{\langle #1 \left|\uparrow\rangle\right.}
\newcommand{\bradown}[1]{\langle #1 \left|\downarrow\rangle\right.}
\newcommand{\av}[1]{\left| #1 \right|}
\newcommand{\op}[1]{\hat{#1}}
\newcommand{\braopket}[3]{\langle #1 | {#2} | #3 \rangle}
\newcommand{\ketbra}[2]{\ket{#1}\bra{#2}}
\newcommand{\pp}[1]{\frac{\partial}{\partial #1}}
\newcommand{\ppn}[1]{\frac{\partial^2}{\partial #1^2}}
\newcommand{\up}{\left|\uparrow\rangle\right.}
\newcommand{\upup}{\left|\uparrow\uparrow\rangle\right.}
\newcommand{\down}{\left|\downarrow\rangle\right.}
\newcommand{\downdown}{\left|\downarrow\downarrow\rangle\right.}
\newcommand{\updown}{\left|\uparrow\downarrow\rangle\right.}
\newcommand{\downup}{\left|\downarrow\uparrow\rangle\right.}
\newcommand{\bupup}{\left.\langle\uparrow\uparrow\right|}
\newcommand{\bdowndown}{\left.\langle\downarrow\downarrow\right|}
\newcommand{\bupdown}{\left.\langle\uparrow\downarrow\right|}
\newcommand{\bdownup}{\left.\langle\downarrow\uparrow\right|}
\renewcommand{\d}{{\rm d}}
\renewcommand{\b}{\bigg}
\newcommand{\Res}[2]{{\rm Res}(#1;#2)}
\newcommand{\To}{\quad\Rightarrow\quad}
\newcommand{\eps}{\epsilon}
\newcommand{\inner}[2]{\langle #1 , #2 \rangle}


\newcommand{\bt}[1]{\boldsymbol{#1}}
\newcommand{\mat}[1]{\textsf{\textbf{#1}}}
\newcommand{\I}{\boldsymbol{\mathcal{I}}}
\newcommand{\p}{\partial}
%
% Navn og tittel
%
\author{}
\title{Notes for project 1 in FYS4460}


\begin{document}

\section*{Units}

We will measure distance in units of $\sigma$, so
$$\bar{r} = \frac{r}{\sigma}.$$

We measure energy in units of $\eps$, so we have
$$\bar{U} = \frac{U}{\eps}.$$

The force is given as the derivative of the potential
$$F = -\frac{\d U}{\d r},$$
substituting $\bar U$ and $\bar r$ gives
$$F = \bigg(\frac{\eps}{\sigma}\bigg)\bigg(-\frac{\d \bar{U}}{\d \bar{r}}\bigg) = \bigg(\frac{\eps}{\sigma}\bigg)\bar{F},$$
so we see that
$$\bar{F} = \bigg(\frac{\sigma}{\eps}\bigg)F.$$

We want Newton's 2.\ law to simplify to
$$\bar{a} = \bar{F},$$
let us see what this means. From the definition of acceleration we get
$$a = \frac{\d^2 r}{\d t^2} = \frac{\sigma}{t_0^2} \frac{\d^2 \bar{r}}{d\bar{t}^2} = \frac{\sigma}{t_0^2}\bar{a}.$$ 
So we have
$$a = \frac{F}{m} \qquad \Rightarrow \qquad  \frac{\sigma}{t_0^2}\bar{a} = \bigg(\frac{\eps}{\sigma}\bigg)\frac{\bar{F}}{m} \qquad \Rightarrow \qquad \frac{\sigma}{t_0^2} = \bigg(\frac{\eps}{m\sigma}\bigg).$$
Solving for $t$ gives
$$t_0 = \sigma\sqrt{\frac{m}{\eps}}.$$
Which will be our unit of time.

This means we find the kinetic energy from
$$K = \frac{1}{2}mv^2 = \frac{1}{2}\frac{\sigma}{mt_0^2}\bar{v}^2 = \frac{1}{2}\eps \bar{v}^2 \quad \Rightarrow \quad \bar{K} = \frac{1}{2}\bar{V}^2.$$



\end{document}



