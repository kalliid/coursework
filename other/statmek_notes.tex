\documentclass[a4paper, 11pt, notitlepage, english]{article}

\usepackage{babel}
\usepackage[utf8]{inputenc}
\usepackage[T1]{fontenc, url}
\usepackage{textcomp}
\usepackage{amsmath, amssymb}
\usepackage{amsbsy, amsfonts}
\usepackage{graphicx, color}
\usepackage{parskip}
\usepackage{framed}
\usepackage{amsmath}
\usepackage{xcolor}
\usepackage{multicol}
\usepackage{url}
\usepackage{flafter}
\usepackage{caption}

%\DeclareCaptionLabelSeparator{colon}{. }
\renewcommand{\captionfont}{\sffamily}
\renewcommand{\captionlabelfont}{\bf\sffamily}
\setlength{\captionmargin}{40pt}

\usepackage{geometry}
\geometry{headheight=0.01mm}
\geometry{top=24mm, bottom=29mm, left=39mm, right=39mm}

\renewcommand{\arraystretch}{2}
\setlength{\tabcolsep}{10pt}
\makeatletter
\renewcommand*\env@matrix[1][*\c@MaxMatrixCols c]{%
  \hskip -\arraycolsep
  \let\@ifnextchar\new@ifnextchar
  \array{#1}}
%
% Parametere for inkludering av kode fra fil
%
\usepackage{listings}

\definecolor{javared}{rgb}{0.6,0,0} % for strings
\definecolor{javagreen}{rgb}{0.25,0.5,0.35} % comments
\definecolor{javapurple}{rgb}{0.5,0,0.35} % keywords
\definecolor{javadocblue}{rgb}{0.25,0.35,0.75} % javadoc

\lstset{language=python,
basicstyle=\ttfamily\scriptsize,
keywordstyle=\color{javapurple},%\bfseries,
stringstyle=\color{javared},
commentstyle=\color{javagreen},
morecomment=[s][\color{javadocblue}]{/**}{*/},
% numbers=left,
% numberstyle=\tiny\color{black},
stepnumber=2,
numbersep=10pt,
tabsize=4,
showspaces=false,
captionpos=b,
showstringspaces=false,
frame= single,
breaklines=true}

%
% Definering av egne kommandoer og miljøer
%
\newcommand{\dd}[1]{\ \text{d}#1}
\newcommand{\f}[2]{\frac{#1}{#2}} 
\newcommand{\beq}{\begin{equation}}
\newcommand{\eeq}{\end{equation}}
\newcommand{\bra}[1]{\langle #1|}
\newcommand{\ket}[1]{|#1 \rangle}
\newcommand{\braket}[2]{\langle #1 | #2 \rangle}
\newcommand{\braup}[1]{\langle #1 \left|\uparrow\rangle\right.}
\newcommand{\bradown}[1]{\langle #1 \left|\downarrow\rangle\right.}
\newcommand{\av}[1]{\left| #1 \right|}
\newcommand{\op}[1]{\hat{#1}}
\newcommand{\braopket}[3]{\langle #1 | {#2} | #3 \rangle}
\newcommand{\ketbra}[2]{\ket{#1}\bra{#2}}
\newcommand{\pp}[1]{\frac{\partial}{\partial #1}}
\newcommand{\ppn}[1]{\frac{\partial^2}{\partial #1^2}}
\newcommand{\up}{\left|\uparrow\rangle\right.}
\newcommand{\upup}{\left|\uparrow\uparrow\rangle\right.}
\newcommand{\down}{\left|\downarrow\rangle\right.}
\newcommand{\downdown}{\left|\downarrow\downarrow\rangle\right.}
\newcommand{\updown}{\left|\uparrow\downarrow\rangle\right.}
\newcommand{\downup}{\left|\downarrow\uparrow\rangle\right.}
\newcommand{\bupup}{\left.\langle\uparrow\uparrow\right|}
\newcommand{\bdowndown}{\left.\langle\downarrow\downarrow\right|}
\newcommand{\bupdown}{\left.\langle\uparrow\downarrow\right|}
\newcommand{\bdownup}{\left.\langle\downarrow\uparrow\right|}
\renewcommand{\d}{{\rm d}}
\newcommand{\Res}[2]{{\rm Res}(#1;#2)}
\newcommand{\To}{\quad\Rightarrow\quad}
\newcommand{\eps}{\epsilon}
\newcommand{\inner}[2]{\langle #1 , #2 \rangle}


\newcommand{\bt}[1]{\boldsymbol{#1}}
\newcommand{\mat}[1]{\textsf{\textbf{#1}}}
\newcommand{\I}{\boldsymbol{\mathcal{I}}}
\newcommand{\p}{\partial}
%
% Navn og tittel
%
\author{Jonas van den Brink \\ \texttt{j.v.d.brink@fys.uio.no}}
\title{MAT-INF3360 \\ Mandatory Exercises 1}

\begin{document}

We are studying a stochastic variabel
$$X = \sum_{i=1}^N x_i,$$
where the $x_i$'s are random, independant, samples of a distribution
$$P(x) \propto \frac{1}{x}.$$

\subsection*{2.)}
If we let $y$ be a sample from the uniform distribution
$$P_0(y) \sim \mbox{uniform}(-1,1) = \begin{cases}
    1/2 & \mbox{if } x\in[-1,1] \\
    0 & \mbox{else},
\end{cases}$$
and we let $x$ be a dependant variable of $y$, meaning it is given from some function
$x(y)$. If we then denote the distribution of $x$ as $P(x)$, we know that
$$P(x) \d x = P_0(y)  \d y$$
If we divide by $\d y$ we get
$$P(x) \frac{\d x}{\d y} = P_0(y),$$
we can now divide by $\d x/\d y = x'(y)$ to show that
$$P(x) = \frac{P_0(y)}{x'(y)}.$$

\subsection*{3.)}
We now let $x = Be^{Ay}$, and want to find $P(x)$, we first find the derivative
$$\frac{1}{x'(y)} = \frac{1}{Ax},$$
giving
$$P(x) = \begin{cases}
    \frac{1}{2Ax} & \mbox{if }  Be^{-A} < x < Be^A \\
    0 & \mbox{else}
\end{cases} $$

\subsection*{4.)}
We check that $P(x)$ is normalized
$$\int_{-\infty}^\infty P(x) \ \d x = \int_{Be^{-A}}^{Be^{A}} \frac{1}{2Ax}\ \d x = \frac{1}{2A}\ln(x)\bigg|_{Be^{-A}}^{Be^{A}} = \frac{1}{2A}\bigg(\ln\big(Be^A\big) - \ln\big({Be^{-A}}\big)\bigg) = 1.$$
And find the expectation value
$$\langle x \rangle = \int_{-\infty}^\infty x P(x) \ \d x = \int_{Be^{-A}}^{Be^{A}} \frac{1}{2A}\ \d x = \frac{B}{A}\frac{e^A - e^{-A}}{2} = \frac{B}{A}\sinh(A).$$
and
$$\langle x^2 \rangle = \int_{-\infty}^\infty x^2 P(x) \ \d x = \int_{Be^{-A}}^{Be^{A}} \frac{x}{2A}\ \d x = \frac{B^2}{2A}\frac{e^{2A} - e^{-2A}}{2} = \frac{B^2}{2A}\sinh(2A).$$
We can now use the trigonometric identity
$$\sinh(2A) = 2\sinh(A)\cosh(A),$$
to find
$$\langle x^2 \rangle = \int_{-\infty}^\infty x^2 P(x) \ \d x = \int_{Be^{-A}}^{Be^{A}} \frac{x}{2A}\ \d x = \frac{B^2}{2A}\frac{e^{2A} - e^{-2A}}{2} = \frac{B^2}{A}\sinh(A)\cosh(A).$$
So the variance is then
$$\langle \Delta x^2 \rangle = \langle x ^2\rangle - \langle x \rangle^2 = \big(A\mbox{ cotanh}(A)-1\big)\frac{B^2}{A^2}\sinh^2(A).$$

\end{document}
