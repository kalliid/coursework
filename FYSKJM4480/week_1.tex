\documentclass[a4paper, 11pt, notitlepage, english]{article}

\usepackage{babel}
\usepackage[utf8]{inputenc}
\usepackage[T1]{fontenc, url}
\usepackage{textcomp}
\usepackage{amsmath, amssymb}
\usepackage{amsbsy, amsfonts}
\usepackage{graphicx, color}
\usepackage{parskip}
\usepackage{framed}
\usepackage{amsmath}
\usepackage{xcolor}
\usepackage{multicol}
\usepackage{url}
\usepackage{flafter}


\usepackage{geometry}
\geometry{headheight=0.01mm}
\geometry{top=24mm, bottom=29mm, left=39mm, right=39mm}

\renewcommand{\arraystretch}{2}
\setlength{\tabcolsep}{10pt}
\makeatletter
\renewcommand*\env@matrix[1][*\c@MaxMatrixCols c]{%
  \hskip -\arraycolsep
  \let\@ifnextchar\new@ifnextchar
  \array{#1}}
%
% Parametere for inkludering av kode fra fil
%
\usepackage{listings}
\lstset{language=python}
\lstset{basicstyle=\ttfamily\small}
\lstset{frame=single}
\lstset{keywordstyle=\color{red}\bfseries}
\lstset{commentstyle=\itshape\color{blue}}
\lstset{showspaces=false}
\lstset{showstringspaces=false}
\lstset{showtabs=false}
\lstset{breaklines}

%
% Definering av egne kommandoer og miljøer
%
\newcommand{\dd}[1]{\ \text{d}#1}
\newcommand{\f}[2]{\frac{#1}{#2}} 
\newcommand{\beq}{\begin{equation}}
\newcommand{\eeq}{\end{equation}}
\newcommand{\bra}[1]{\langle #1|}
\newcommand{\ket}[1]{|#1 \rangle}
\newcommand{\braket}[2]{\langle #1 | #2 \rangle}
\newcommand{\braup}[1]{\langle #1 \left|\uparrow\rangle\right.}
\newcommand{\bradown}[1]{\langle #1 \left|\downarrow\rangle\right.}
\newcommand{\av}[1]{\left| #1 \right|}
\newcommand{\op}[1]{\hat{#1}}
\newcommand{\braopket}[3]{\langle #1 | {#2} | #3 \rangle}
\newcommand{\ketbra}[2]{\ket{#1}\bra{#2}}
\newcommand{\pp}[1]{\frac{\partial}{\partial #1}}
\newcommand{\ppn}[1]{\frac{\partial^2}{\partial #1^2}}
\newcommand{\up}{\left|\uparrow\rangle\right.}
\newcommand{\upup}{\left|\uparrow\uparrow\rangle\right.}
\newcommand{\down}{\left|\downarrow\rangle\right.}
\newcommand{\downdown}{\left|\downarrow\downarrow\rangle\right.}
\newcommand{\updown}{\left|\uparrow\downarrow\rangle\right.}
\newcommand{\downup}{\left|\downarrow\uparrow\rangle\right.}
\newcommand{\bupup}{\left.\langle\uparrow\uparrow\right|}
\newcommand{\bdowndown}{\left.\langle\downarrow\downarrow\right|}
\newcommand{\bupdown}{\left.\langle\uparrow\downarrow\right|}
\newcommand{\bdownup}{\left.\langle\downarrow\uparrow\right|}
\renewcommand{\d}{{\rm d}}
\newcommand{\Res}[2]{{\rm Res}(#1;#2)}
\newcommand{\To}{\quad\Rightarrow\quad}
\newcommand{\eps}{\epsilon}
\newcommand{\inner}[2]{\langle #1 , #2 \rangle}


\newcommand{\bt}[1]{\boldsymbol{#1}}
\newcommand{\mat}[1]{\textsf{\textbf{#1}}}
\newcommand{\I}{\boldsymbol{\mathcal{I}}}
\newcommand{\p}{\partial}
%
% Navn og tittel
%
\author{Jonas van den Brink \\ \texttt{j.v.brink@fys.uio.no}}
\title{Problem set 1 \\ FYS-KJM4480}

\begin{document}
\maketitle

\section*{Exercise 1}

In this problem we consider the Slater determinant
$$\Phi_\lambda^{AS} (x_1, x_2, \ldots, x_N; \alpha_1, \ldots, \alpha_N) = \frac{1}{\sqrt{N!}}\sum_p (-)^p \op{P} \prod_{i=1}^N \psi_{\alpha_i}(x_i).$$
Where $\alpha_i$ are quantum numbers and $N$ is the number of particles. The sum over $p$ is a summation over all possible permutations.

\subsection*{a)}
If we let $N=3$, the Slater determinant becomes:
\begin{align*}   
\Phi_\lambda^{AS}(\bt{x};\bt{\alpha}) = \frac{1}{\sqrt{6}} \bigg(
&\psi_{\alpha_1}(x_1) \psi_{\alpha_2} (x_2)\psi_{\alpha_3}(x_3)
- \psi_{\alpha_1}(x_2) \psi_{\alpha_2} (x_1)\psi_{\alpha_3}(x_3) \\
&\quad+ \psi_{\alpha_1}(x_2) \psi_{\alpha_2} (x_3)\psi_{\alpha_3}(x_1)  
- \psi_{\alpha_1}(x_3) \psi_{\alpha_2} (x_2)\psi_{\alpha_3}(x_1) \\
&\qquad+ \psi_{\alpha_1}(x_3) \psi_{\alpha_2} (x_1)\psi_{\alpha_3}(x_2) 
- \psi_{\alpha_1}(x_1) \psi_{\alpha_2} (x_3)\psi_{\alpha_3}(x_2)\bigg)
\end{align*}

\subsection*{b)}
We will no show that the slater determinant is normalized, in the sense that
$$\braket{\Phi_\lambda^{AS}}{\Phi_\lambda^{AS}} = \int |\Phi_\lambda^{AS} (x_1,\ldots, x_N, \alpha_1, \ldots, \alpha_N)|^2 = 1.$$
To do this we will assume that the single-particle states forms an orthonormal set, i.e.,
$$\braket{\psi_{\alpha_i}}{\psi_{\alpha_j}} = \int \psi^*_{\alpha_i}(x)\psi_{\alpha_j}(x) \ \d x = \delta_{ij},$$
where $\delta_{ij}$ is the Kronecker-delta.

To easily see that the slater determinant is normalized, we write it using the antisymmetrizer operator
$$\Phi_\lambda^{AS} = \sqrt{N!}\mathcal{A}\prod_{i=1}^N \phi_H.$$
Where we have used the \emph{Hartree-function}:
$$\phi_H \equiv \prod_{i=1}^N \psi_{\alpha_i}(x_i).$$

We can then write the inner-product as
$$\braket{\Phi_\lambda^{AS}}{\Phi_\lambda^{AS}} = N! \int \mathcal{A}^* \phi^*_H \mathcal{A} \phi_H \ \d \bt{x} .$$
We can simplify this as follows
$$\braket{\Phi_\lambda^{AS}}{\Phi_\lambda^{AS}} = N! \int \phi^*_H \mathcal{A}^2 \phi_H \ \d \bt{x} = N! \int \phi^*_H \mathcal{A} \phi_H \ \d \bt{x}.$$
Where we have used the fact that the antisymmetrizer is unitary, $\mathcal{A}^\dagger = \mathcal{A}$ and that it is a projection operator, meaning $\mathcal{A}^2 = \mathcal{A}$.
We now write out the definition of the antisymmetrizer, giving
$$\braket{\Phi_\lambda^{AS}}{\Phi_\lambda^{AS}} = \sum_p (-)^p \int \phi^*_H \\\op{P} \phi_H \ \d \bt{x}.$$
As the permutation operator only acts on one of the Hartree-functions, the two functions will never be similar and the orthogonality of the single-particle states makes these cross-products cancel out. We are left only with the contribution when the permuation operator is identity, giving
$$\braket{\Phi_\lambda^{AS}}{\Phi_\lambda^{AS}} = \int \phi^*_H \phi_H \ \d \bt{x} = 1.$$
Where we again used our assumption about the normality of the single-particle states.

\subsection*{c)}
We now define a general onebody operator and a general twobody operator:
$$\op{F} = \sum_{i=1}^N \op{f}(x_i), \qquad \op{G} = \sum_{i<j} \op{g}(x_i, x_j).$$
And we will now calculate the expectation value of these operators for a two-particle Slater determinant.
$$\braopket{\Phi^{AS}_{\alpha_1\alpha_2}}{\op{F}}{\Phi^{AS}_{\alpha_1\alpha_2}}, \qquad \braopket{\Phi^{AS}_{\alpha_1\alpha_2}}{\op{G}}{\Phi^{AS}_{\alpha_1\alpha_2}}.$$

We start with the onebody operator, using the same technique as above using the antisymmetrizer, we can write out the integral as follows
$$\braopket{\Phi^{AS}_{\alpha_1\alpha_2}}{\op{F}}{\Phi^{AS}_{\alpha_1\alpha_2}} = N! \int \phi_H^* \op{F} \mathcal{A} \phi_H \ \d \bt{x}.$$
We now insert the definitions of the two operators
$$\braopket{\Phi^{AS}_{\alpha_1\alpha_2}}{\op{F}}{\Phi^{AS}_{\alpha_1\alpha_2}} = \sum_{i=1}^N \sum_p (-)^p \int \phi_H^* \op{f} \op{P} \phi_H \ \d \bt{x}.$$
As earlier, if one of the Hartree-functions is permutated, the integral vanishes, so we are left with
$$\braopket{\Phi^{AS}_{\alpha_1\alpha_2}}{\op{F}}{\Phi^{AS}_{\alpha_1\alpha_2}} = \sum_{i=1}^N \int \phi_H^* \op{f} \phi_H \ \d \bt{x}.$$

\end{document}

