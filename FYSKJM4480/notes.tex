\documentclass[a4paper, 11pt, notitlepage, english]{article}

\usepackage{babel}
\usepackage[utf8]{inputenc}
\usepackage[T1]{fontenc, url}
\usepackage{textcomp}
\usepackage{amsmath, amssymb}
\usepackage{amsbsy, amsfonts}
\usepackage{graphicx, color}
\usepackage{parskip}
\usepackage{framed}
\usepackage{amsmath}
\usepackage{xcolor}
\usepackage{multicol}
\usepackage{url}
\usepackage{flafter}


\usepackage{geometry}
\geometry{headheight=0.01mm}
\geometry{top=24mm, bottom=29mm, left=39mm, right=39mm}

\renewcommand{\arraystretch}{2}
\setlength{\tabcolsep}{10pt}
\makeatletter
\renewcommand*\env@matrix[1][*\c@MaxMatrixCols c]{%
  \hskip -\arraycolsep
  \let\@ifnextchar\new@ifnextchar
  \array{#1}}
%
% Parametere for inkludering av kode fra fil
%
\usepackage{listings}
\lstset{language=python}
\lstset{basicstyle=\ttfamily\small}
\lstset{frame=single}
\lstset{keywordstyle=\color{red}\bfseries}
\lstset{commentstyle=\itshape\color{blue}}
\lstset{showspaces=false}
\lstset{showstringspaces=false}
\lstset{showtabs=false}
\lstset{breaklines}

%
% Definering av egne kommandoer og miljøer
%
\newcommand{\dd}[1]{\ \text{d}#1}
\newcommand{\f}[2]{\frac{#1}{#2}} 
\newcommand{\beq}{\begin{equation}}
\newcommand{\eeq}{\end{equation}}
\newcommand{\bra}[1]{\langle #1|}
\newcommand{\ket}[1]{|#1 \rangle}
\newcommand{\braket}[2]{\langle #1 | #2 \rangle}
\newcommand{\braup}[1]{\langle #1 \left|\uparrow\rangle\right.}
\newcommand{\bradown}[1]{\langle #1 \left|\downarrow\rangle\right.}
\newcommand{\av}[1]{\left| #1 \right|}
\newcommand{\op}[1]{\hat{#1}}
\newcommand{\braopket}[3]{\langle #1 | {#2} | #3 \rangle}
\newcommand{\ketbra}[2]{\ket{#1}\bra{#2}}
\newcommand{\pp}[1]{\frac{\partial}{\partial #1}}
\newcommand{\ppn}[1]{\frac{\partial^2}{\partial #1^2}}
\newcommand{\up}{\left|\uparrow\rangle\right.}
\newcommand{\upup}{\left|\uparrow\uparrow\rangle\right.}
\newcommand{\down}{\left|\downarrow\rangle\right.}
\newcommand{\downdown}{\left|\downarrow\downarrow\rangle\right.}
\newcommand{\updown}{\left|\uparrow\downarrow\rangle\right.}
\newcommand{\downup}{\left|\downarrow\uparrow\rangle\right.}
\newcommand{\bupup}{\left.\langle\uparrow\uparrow\right|}
\newcommand{\bdowndown}{\left.\langle\downarrow\downarrow\right|}
\newcommand{\bupdown}{\left.\langle\uparrow\downarrow\right|}
\newcommand{\bdownup}{\left.\langle\downarrow\uparrow\right|}
\renewcommand{\d}{{\rm d}}
\newcommand{\Res}[2]{{\rm Res}(#1;#2)}
\newcommand{\To}{\quad\Rightarrow\quad}
\newcommand{\eps}{\epsilon}
\newcommand{\inner}[2]{\langle #1 , #2 \rangle}


\newcommand{\bt}[1]{\boldsymbol{#1}}
\newcommand{\mat}[1]{\textsf{\textbf{#1}}}
\newcommand{\I}{\boldsymbol{\mathcal{I}}}
\newcommand{\p}{\partial}
%
% Navn og tittel
%
\author{}
\title{Notes in FYS-KJM4480}


\begin{document}

\begin{center}
{\huge \bf Notes in FYS-KJM4480}   
\end{center}

\vspace{1cm}

We let $\op{P}$ denote an operator which interchanges two particles. The Hamlitonian must be invariant under this change, so we know that the operators commute
$$[\op{H}, \op{P}] = 0,$$
As they commute, they share eigenfunctions, meaning we have
$$\op{P}_{ij}\psi_\lambda(x_1,\ldots, x_N) = \beta \psi_\lambda(x_1,\ldots,x_N).$$
Where $\beta$ will be 1 for bosons and -1 for fermions.

The Hamiltonian is generally
$$\op{H} = \op{T} + \op{V},$$
where $\op{T}$ represents the kinetic energy of the system
$$\op{T} = \sum_{i=1}^N \frac{\bt p_i^2}{2m_i} = \sum_{i}^N\bigg(-\frac{\hbar^2}{2m_i}\bigg) \nabla_i^2 = \sum_{i=1}^N t(x_i).$$
while the potential energy operator is
$$\op{V} = \sum_{i=1}^{N} \op{u}_{ext}(x_i) + \sum_{ji=1}^{N}v(x_i,x_j) + \ldots$$

Using the Born-Oppenheimer approximation means we can look at molecules by looking at the electronic and nuclear degrees of freedom seperately. If we focus only on the electronic degrees of freedom, meaning we freeze out the nuclear dofs, we have
$$\op{H} = \sum_{i=1}^{n_e} t(x_i) - \sum_{i=1}^{n_e} k \frac{Z}{r_i} + \sum_{i<j}^{n_e} \frac{k}{r_{ij}}$$
with $k=1.44$ eV nm.

We can then split the Hamiltonian in to terms, one which is the sum of $N$ one-body Hamiltonians, $\op{h}_0$, and one which is the sum of the $n_e(n_e -1)/2$ two-body interactions.

\section*{Slater Determinant}

$$\Phi^{AS} = \frac{1}{\sqrt{N!}} \sum_p (-)^p \op{P} \prod_{i=1}^N \psi_{\alpha_i}(x_i) = \sqrt{N!}\mathcal{A} \phi_H.$$
Where $\mathcal{A}$ is the antisymmetrization operator 
$$\mathcal{A} = \frac{1}{N!}\sum_p (-)^p \op{P}.$$

\end{document}

