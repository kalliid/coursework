\documentclass[a4paper, 11pt, notitlepage, english]{article}

\usepackage{babel, textcomp}
\usepackage[latin1]{inputenc}
\usepackage[T1]{fontenc, url}
\usepackage{amsmath, amssymb}
\usepackage{amsbsy, amsfonts}
\usepackage{graphicx, color, xcolor}
\usepackage{framed, parskip}
\usepackage{flafter, caption, multicol}
\usepackage{verbatim, listings, fancyvrb}
\usepackage{simplewick}


%\DeclareCaptionLabelSeparator{colon}{. }
\renewcommand{\captionfont}{\small\sffamily}
\renewcommand{\captionlabelfont}{\bf\sffamily}
\setlength{\captionmargin}{20pt}

\setcounter{tocdepth}{2}

\DeclareMathAlphabet{\mathbfit}{OML}{cmm}{b}{it}

\definecolor{javared}{rgb}{0.6,0,0} % for strings
\definecolor{javagreen}{rgb}{0.25,0.5,0.35} % comments
\definecolor{javapurple}{rgb}{0.5,0,0.35} % keywords
\definecolor{javadocblue}{rgb}{0.25,0.35,0.75} % javadoc

\lstset{language=python,
basicstyle=\ttfamily\scriptsize,
keywordstyle=\color{javapurple},%\bfseries,
stringstyle=\color{javared},
commentstyle=\color{javagreen},
morecomment=[s][\color{javadocblue}]{/**}{*/},
morekeywords={super, with},
% numbers=left,
% numberstyle=\tiny\color{black},
stepnumber=2,
numbersep=10pt,
tabsize=4,
showspaces=false,
captionpos=b,
showstringspaces=false,
frame= single,
breaklines=true}
% 
% \makeatletter
% \def\lst@lettertrue{\let\lst@ifletter\iffalse}
% \makeatother

\usepackage{geometry}
\geometry{headheight=0.01mm}
\geometry{top=20mm, bottom=20mm, left=34mm, right=34mm}



\renewcommand{\arraystretch}{2}
\setlength{\tabcolsep}{10pt}
\makeatletter
\renewcommand*\env@matrix[1][*\c@MaxMatrixCols c]{%
  \hskip -\arraycolsep
  \let\@ifnextchar\new@ifnextchar
  \array{#1}}
\makeatother



%
% Definering av egne kommandoer og miljøer
%
\newcommand{\dd}[1]{\ \text{d}#1}
\newcommand{\f}[2]{\frac{#1}{#2}} 
\newcommand{\beq}{\begin{equation}}
\newcommand{\eeq}{\end{equation}}
\newcommand{\bra}[1]{\langle #1|}
\newcommand{\ket}[1]{|#1 \rangle}
\newcommand{\braket}[2]{\langle #1 | #2 \rangle}
\newcommand{\braup}[1]{\langle #1 \left|\uparrow\rangle\right.}
\newcommand{\bradown}[1]{\langle #1 \left|\downarrow\rangle\right.}
\newcommand{\av}[1]{\left| #1 \right|}
\newcommand{\op}[1]{\hat{#1}}
\newcommand{\braopket}[3]{\langle #1 | {#2} | #3 \rangle}
\newcommand{\ketbra}[2]{\ket{#1}\bra{#2}}
\newcommand{\pp}[1]{\frac{\partial}{\partial #1}}
\newcommand{\ppn}[1]{\frac{\partial^2}{\partial #1^2}}
\newcommand{\up}{\left|\uparrow\rangle\right.}
\newcommand{\upup}{\left|\uparrow\uparrow\rangle\right.}
\newcommand{\down}{\left|\downarrow\rangle\right.}
\newcommand{\downdown}{\left|\downarrow\downarrow\rangle\right.}
\newcommand{\updown}{\left|\uparrow\downarrow\rangle\right.}
\newcommand{\downup}{\left|\downarrow\uparrow\rangle\right.}
\newcommand{\bupup}{\left.\langle\uparrow\uparrow\right|}
\newcommand{\bdowndown}{\left.\langle\downarrow\downarrow\right|}
\newcommand{\bupdown}{\left.\langle\uparrow\downarrow\right|}
\newcommand{\bdownup}{\left.\langle\downarrow\uparrow\right|}
\renewcommand{\d}{{\rm d}}
\newcommand{\Res}[2]{{\rm Res}(#1;#2)}
\newcommand{\To}{\quad\Rightarrow\quad}
\newcommand{\eps}{\epsilon}
\newcommand{\inner}[2]{\langle #1 , #2 \rangle}


\newcommand{\bt}[1]{\boldsymbol{#1}}
\newcommand{\mat}[1]{\textsf{\textbf{#1}}}
\newcommand{\I}{\boldsymbol{\mathcal{I}}}
\newcommand{\p}{\partial}
%
% Navn og tittel
%
\author{Jonas van den Brink \\ \texttt{j.v.brink@fys.uio.no}}
\title{Problem set 3 \\ FYS-KJM4480}

\begin{document}
\pagestyle{empty}

The solution for exercise 7b is given as
\begin{equation}
\op{H}_I = \frac{1}{4}\sum_{pqrs}\braopket{pq}{\op{v}}{rs}\{\op{a}_q^\dag \op{a}_p^\dag \op{a}_r \op{a}_s\} + \sum_{pqi}\braopket{pi}{\op{v}}{qi}\{\op{a}_p^\dag\op{a}_p\} + \frac{1}{2}\sum_{ij}	\braopket{ij}{\op{v}}{ij}.
\end{equation}
Where it is stated that the double sum carries a restriction $i<j$.

In Shavitt and Bartlett (p.\ 82, eq.\ 3.169) however, the answer is given as
\begin{equation}
\op{H}_I = \frac{1}{4}\sum_{pqrs}\braopket{pq}{\op{v}}{rs}_{\rm A}\{\op{a}_q^\dag \op{a}_p^\dag \op{a}_r \op{a}_s\} + \sum_{pqi}\braopket{pi}{\op{v}}{qi}_{\rm A}\{\op{a}_p^\dag\op{a}_p\} + \frac{1}{2}\sum_{ij}	\braopket{ij}{\op{v}}{ij}_{\rm A}.
\end{equation}
With no restriction on the double sum.

There are two differences between (1) and (2). First, in (2) all the matrix elements are asymmetrized, denoted by the subindex A. However, in the lectures, it has been mentioned that this subindex will be dropped, and it is left to the reader to infer from the 1/4 fraction that the matrix elements should be interpreted as asymmetrized. Now, the first term carries the 1/4 fraction, but the latter terms does not. As we started from the asymmetrized term, it is apparant that all three terms in (1) actually are asymmetrized, altough I personally think this should be reflected in the notation, as this can lead to a lot of confusion.

The second difference is much bigger, as (1) carries a restriction $i<j$ on the double sum, while (2) has no such restriction. If we let the Fermi vacuum be a simple 2-particle system: $\ket{\alpha\beta}$, let us write out the expectation value of the interacting part of the Hamiltonian for the Fermi vacuum
\paragraph{Equation (1)}
\begin{align*}
\braopket{\Phi_0}{\op{H}_I}{\Phi_0} &= \frac{1}{2}\sum_{i<j} \braopket{ij}{\op{v}}{ij}_{\rm A} = \frac{1}{2}\braopket{\alpha\beta}{\op{v}}{\alpha\beta}_{\rm A}.
\end{align*}
\paragraph{Equation (2)}
$$\braopket{\Phi_0}{\op{H}_I}{\Phi_0} =  \frac{1}{2}\sum_{ij} \braopket{ij}{\op{v}}{ij}_{\rm A} = \frac{1}{2}\bigg(\braopket{\alpha\alpha}{\op{v}}{\alpha\alpha}_{\rm A} + \braopket{\alpha\beta}{\op{v}}{\alpha\beta}_{\rm A} + \braopket{\beta\alpha}{\op{v}}{\beta\alpha}_{\rm A} + \braopket{\beta\beta}{\op{v}}{\beta\beta}_{\rm A}\bigg).$$
Using 
$$\braopket{ii}{\op{v}}{jj}_{\rm A} = \braopket{ii}{\op{v}}{jj} - \braopket{ii}{\op{v}}{jj} = 0,$$
and
$$\braopket{ij}{\op{v}}{ij}_{\rm A} = \braopket{ij}{\op{v}}{ij} - \braopket{ij}{\op{v}}{ji} = \braopket{ji}{\op{v}}{ji} - \braopket{ji}{\op{v}}{ij} = \braopket{ji}{\op{v}}{ji}_{\rm A} $$
Which gives
$$\braopket{\Phi_0}{\op{H}_I}{\Phi_0} = \braopket{\alpha\beta}{\op{v}}{\alpha\beta}_{\rm A}.$$

It is clearly equation (2) which gives the right result. And so there should not be any $i<j$ restriction. Another solution can be to remove the factor 1/2 in front, which arises because we count all pair of particles twice, if we require $i<j$, we are \emph{not} counting all pairs twice.



\end{document}