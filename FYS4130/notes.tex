\documentclass[a4paper, 11pt, notitlepage, english]{article}

\usepackage{babel}
\usepackage[utf8]{inputenc}
\usepackage[T1]{fontenc, url}
\usepackage{textcomp}
\usepackage{amsmath, amssymb}
\usepackage{amsbsy, amsfonts}
\usepackage{graphicx, color}
\usepackage{parskip}
\usepackage{framed}
\usepackage{amsmath}
\usepackage{xcolor}
\usepackage{multicol}
\usepackage{url}
\usepackage{flafter}


\usepackage{geometry}
\geometry{headheight=0.01mm}
\geometry{top=24mm, bottom=29mm, left=39mm, right=39mm}

\renewcommand{\arraystretch}{2}
\setlength{\tabcolsep}{10pt}
\makeatletter
\renewcommand*\env@matrix[1][*\c@MaxMatrixCols c]{%
  \hskip -\arraycolsep
  \let\@ifnextchar\new@ifnextchar
  \array{#1}}
%
% Parametere for inkludering av kode fra fil
%
\usepackage{listings}
\lstset{language=python}
\lstset{basicstyle=\ttfamily\small}
\lstset{frame=single}
\lstset{keywordstyle=\color{red}\bfseries}
\lstset{commentstyle=\itshape\color{blue}}
\lstset{showspaces=false}
\lstset{showstringspaces=false}
\lstset{showtabs=false}
\lstset{breaklines}

%
% Definering av egne kommandoer og miljøer
%
\newcommand{\dd}[1]{\ \text{d}#1}
\newcommand{\f}[2]{\frac{#1}{#2}} 
\newcommand{\beq}{\begin{equation}}
\newcommand{\eeq}{\end{equation}}
\newcommand{\bra}[1]{\langle #1|}
\newcommand{\ket}[1]{|#1 \rangle}
\newcommand{\braket}[2]{\langle #1 | #2 \rangle}
\newcommand{\braup}[1]{\langle #1 \left|\uparrow\rangle\right.}
\newcommand{\bradown}[1]{\langle #1 \left|\downarrow\rangle\right.}
\newcommand{\av}[1]{\left| #1 \right|}
\newcommand{\op}[1]{\hat{#1}}
\newcommand{\braopket}[3]{\langle #1 | {#2} | #3 \rangle}
\newcommand{\ketbra}[2]{\ket{#1}\bra{#2}}
\newcommand{\pp}[1]{\frac{\partial}{\partial #1}}
\newcommand{\ppn}[1]{\frac{\partial^2}{\partial #1^2}}
\newcommand{\up}{\left|\uparrow\rangle\right.}
\newcommand{\upup}{\left|\uparrow\uparrow\rangle\right.}
\newcommand{\down}{\left|\downarrow\rangle\right.}
\newcommand{\downdown}{\left|\downarrow\downarrow\rangle\right.}
\newcommand{\updown}{\left|\uparrow\downarrow\rangle\right.}
\newcommand{\downup}{\left|\downarrow\uparrow\rangle\right.}
\newcommand{\bupup}{\left.\langle\uparrow\uparrow\right|}
\newcommand{\bdowndown}{\left.\langle\downarrow\downarrow\right|}
\newcommand{\bupdown}{\left.\langle\uparrow\downarrow\right|}
\newcommand{\bdownup}{\left.\langle\downarrow\uparrow\right|}
\renewcommand{\d}{{\rm d}}
\renewcommand{\b}{\bigg}
\newcommand{\Res}[2]{{\rm Res}(#1;#2)}
\newcommand{\To}{\quad\Rightarrow\quad}
\newcommand{\eps}{\epsilon}
\newcommand{\inner}[2]{\langle #1 , #2 \rangle}


\newcommand{\bt}[1]{\boldsymbol{#1}}
\newcommand{\mat}[1]{\textsf{\textbf{#1}}}
\newcommand{\I}{\boldsymbol{\mathcal{I}}}
\newcommand{\p}{\partial}
%
% Navn og tittel
%
\author{}
\title{Notes in FYS4130---Statistical physics}


\begin{document}

\section*{Laws of Thermodynamics}

\subsubsection*{First law}
The first law is a statement about conservation of energy. If we let $U$ denote the internal energy of a system, we state that we can decompose this energy change as \emph{heat}, $Q$ and \emph{work}, $W$.
$$\Delta U = Q \pm W.$$
Note that we use $\pm$ to denote the work, this is because we can define the work both \emph{on} the system or \emph{by} the system, both conventions are used regurarly and so care should be taken. The heat is usually defined as the heat \emph{entering} the system.

If the number of particles in a system is constant the instantaneous work done by the system will be 
$$\d W = P \d V,$$
so we have
$$\d U = \d Q - P \d V.$$

\subsection*{Second law}
The second law states that the entropy of an isolated system always will increase. Mathematically we can state it as
$$\Delta S \geq T\Delta Q,$$
at least for a constant temperature. For a reversible process, this becomes an equality
$$\Delta S = T \Delta Q \quad \mbox{(reversible process)} $$
This implies that a reversible process is a process that produces no entropy, it is thus a system continously changing through equilibria states and will therefore often be 
infinitely slow. A reversible process is thus ususally an abstraction, and not a real quality of a process. Processes can however, be close to reversible.

Combining the first and second law gives us the important inequality
$$T\Delta S \geq \Delta U + P\Delta V.$$
For an infinitesimal quantity, this will always be true
$$T\ \d S = \d U + P\ \d V.$$

\clearpage

\section*{Maxwell relations}
From the first law we have
\beq \d U = T\ \d S - P\ \d V. \eeq \label{eq:maxwell1}
Should therefore express $U$ as a function of $S$ and $V$, $U = U(S,V)$, can then look at the total derivative of $U$
\beq \d U = \bigg(\frac{\p U}{\p S}\bigg)_V \d S + \bigg(\frac{\p U}{\p V}\bigg)_S \d V. \eeq \label{eq:maxwell2}
Comparing \ref{eq:maxwell1} and \ref{eq:maxwell2} gives
$$T = \bigg(\frac{\p U}{\p S}\bigg)_V, \qquad P = -\bigg(\frac{\p U}{\p V}\bigg)_S.$$

Generally
$$\frac{\p^2 U}{\p V \p S} = \frac{\p^2 U}{\p S \p V}.$$
So we get
$$\bigg(\frac{\p T}{\p V}\bigg)_S = -\bigg(\frac{\p P}{\p S}\bigg)_V.$$

\subsubsection*{General state variables}
Let us say we have three state variables, $X$, $Y$, $Z$, and we have an equation of state $Z(X,Y)$. We see that $Z$ is not a free variable, but is given by $X$ and $Y$. Of course, we could have said that $X$ is not the free variable, because it can be given by $X(Y,Z)$, i.e., the equation of state can be solved for any of the three state variables.

The total derivatives of all equation of states become
\begin{align*}
\d Z = \bigg(\frac{\p Z}{\p X}\bigg)_Y \d X + \bigg(\frac{\p Z}{\p Y}\bigg)_X \d Y, \\
\d X = \bigg(\frac{\p X}{\p Y}\bigg)_Z \d Y + \bigg(\frac{\p X}{\p Z}\bigg)_Y \d Z, \\
\d Y = \bigg(\frac{\p Y}{\p X}\bigg)_Z \d X + \bigg(\frac{\p Y}{\p Z}\bigg)_X \d Z.
\end{align*}
Inserting $\d Y$ into $\d X$ gives
$$\bigg[\bigg(\frac{\p X}{\p Y}\bigg)_Z\bigg(\frac{\p Y}{\p X}\bigg)_Z - 1\bigg]\d X  + \bigg[\bigg(\frac{\p X}{\p Y}\bigg)_Z\bigg(\frac{\p Y}{\p Z}\bigg)_X + \bigg(\frac{\p X}{\p Z}\bigg)_Y\bigg]\d Z = 0.$$
The differentials are of course independant, so we get
$$\bigg(\frac{\p X}{\p Y}\bigg)_Z = \bigg(\frac{\p Y}{\p X}\bigg)_Z^{-1},$$
and
$$\bigg(\frac{\p X}{\p Y}\bigg)_Z \bigg(\frac{\p Y}{\p Z}\bigg)_X \bigg(\frac{\p Z}{\p X}\bigg)_Y = -1.$$

We can now let another state variable be given by $X$ and $Y$
$$\d U = \bigg(\frac{\p U}{\p X}\bigg)_Y \d X + \bigg(\frac{\p U}{\p Y}\bigg)_X \d Y.$$
Dividing by $\d X$ and holding the equation constant at $Z$ gives
$$\b( \frac{\p U}{\p X} \b)_Z = \bigg(\frac{\p U}{\p X}\b)_Y + \b(\frac{\p U}{\p Y}\b)_X \b(\frac{\p Y}{\p X}\b)_Z$$

\subsection*{Specific Heat}
When we add a small amount of heat to a system, the temprature will rise by some small amount, this is the definition of the specific heat of that system. For a given state variable $X$ it is defined as
$$C_X = \lim_{\Delta T \to 0} \b(\frac{\Delta Q}{\Delta T}\b)_X = T \b(\frac{\p S}{\p T}\b)_X.$$

Starting at the first law we have
$$T \d S = \d U + P\ \d V,$$
inserting the complete derivative for $\d U$ gives
$$T \d S = \b(\frac{\p U}{\p T}\b)_V \d T + \b(\frac{\p U}{\p V}\b)_T \d V + P\ \d V,$$
If we look at the specific heat for a constant volume, we get
$$C_V = \b(\frac{\p U}{\p T}\b)_V.$$
If we instead keep the pressure constant, we get
$$C_P = C_V + \b[\b(\frac{\p U}{\p V}\b)_T + P\b]\ \b(\frac{\p V}{\p T}\b)_P.$$
This can also be formulated as
$$C_P = \b(\frac{\p H}{\p T}\b)_P.$$
Where $H$ is the \emph{enthalpy}:
$$H = U + PV.$$

\section*{Equation of State}
For most systems all state variables are not independent, but will depend on each other. If we can explicitly give a state variable in terms of the other state variables, that is an equation of state. An example could be the relation $$P(T,V).$$
 
The equation of state give important information about how matter will behave under different physical conditions and so is very material dependant.

\subsection*{Ideal gas law}
The ideal gas law is an equation of state
$$P = \frac{NkT}{V}, \qquad P = kT\rho.$$

\subsection*{Van der Waals Equation of State}
The ideal gas presupposes no interaction between the particles. The van der Waals equation of state tries to include some interaction in a basic and approximate manner
$$P = \frac{NkT}{V - Nb} - \frac{aN^2}{V^2}.$$
Here $a$ and $b$ are parameters describing the particles, $b$ reflects the particles having some volume and $a$ models the attraction between the particles.




\section*{Constants}

\begin{center}
\begin{tabular}{|c|c|}	
\hline
Boltzmann's constant & $k = 1.381\times10^{-23} \mbox{JK}^{-1}$ \\ \hline 
Avogadro's Number & $N_{\rm A} = 6.023\times10^{23} \mbox{mol}^{-1}$ \\ \hline 
$k\cdot N_{\rm A}$ & $R = 8.314 \mbox{JK}^{-1}\mbox{mol}^{-1}$ \\ \hline
\end{tabular}
\end{center}

\end{document}


