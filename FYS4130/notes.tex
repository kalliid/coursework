\documentclass[a4paper, 11pt, notitlepage, english]{article}

\usepackage{babel}
\usepackage[utf8]{inputenc}
\usepackage[T1]{fontenc, url}
\usepackage{textcomp}
\usepackage{amsmath, amssymb}
\usepackage{amsbsy, amsfonts}
\usepackage{graphicx, color}
\usepackage{parskip}
\usepackage{framed}
\usepackage{amsmath}
\usepackage{xcolor}
\usepackage{multicol}
\usepackage{url}
\usepackage{flafter}


\usepackage{geometry}
\geometry{headheight=0.01mm}
\geometry{top=24mm, bottom=29mm, left=39mm, right=39mm}

\renewcommand{\arraystretch}{2}
\setlength{\tabcolsep}{10pt}
\makeatletter
\renewcommand*\env@matrix[1][*\c@MaxMatrixCols c]{%
  \hskip -\arraycolsep
  \let\@ifnextchar\new@ifnextchar
  \array{#1}}
%
% Parametere for inkludering av kode fra fil
%
\usepackage{listings}
\lstset{language=python}
\lstset{basicstyle=\ttfamily\small}
\lstset{frame=single}
\lstset{keywordstyle=\color{red}\bfseries}
\lstset{commentstyle=\itshape\color{blue}}
\lstset{showspaces=false}
\lstset{showstringspaces=false}
\lstset{showtabs=false}
\lstset{breaklines}

%
% Definering av egne kommandoer og miljøer
%
\newcommand{\dd}[1]{\ \text{d}#1}
\newcommand{\f}[2]{\frac{#1}{#2}} 
\newcommand{\beq}{\begin{equation}}
\newcommand{\eeq}{\end{equation}}
\newcommand{\bra}[1]{\langle #1|}
\newcommand{\ket}[1]{|#1 \rangle}
\newcommand{\braket}[2]{\langle #1 | #2 \rangle}
\newcommand{\braup}[1]{\langle #1 \left|\uparrow\rangle\right.}
\newcommand{\bradown}[1]{\langle #1 \left|\downarrow\rangle\right.}
\newcommand{\av}[1]{\left| #1 \right|}
\newcommand{\op}[1]{\hat{#1}}
\newcommand{\braopket}[3]{\langle #1 | {#2} | #3 \rangle}
\newcommand{\ketbra}[2]{\ket{#1}\bra{#2}}
\newcommand{\pp}[1]{\frac{\partial}{\partial #1}}
\newcommand{\ppn}[1]{\frac{\partial^2}{\partial #1^2}}
\newcommand{\up}{\left|\uparrow\rangle\right.}
\newcommand{\upup}{\left|\uparrow\uparrow\rangle\right.}
\newcommand{\down}{\left|\downarrow\rangle\right.}
\newcommand{\downdown}{\left|\downarrow\downarrow\rangle\right.}
\newcommand{\updown}{\left|\uparrow\downarrow\rangle\right.}
\newcommand{\downup}{\left|\downarrow\uparrow\rangle\right.}
\newcommand{\bupup}{\left.\langle\uparrow\uparrow\right|}
\newcommand{\bdowndown}{\left.\langle\downarrow\downarrow\right|}
\newcommand{\bupdown}{\left.\langle\uparrow\downarrow\right|}
\newcommand{\bdownup}{\left.\langle\downarrow\uparrow\right|}
\renewcommand{\d}{{\rm d}}
\renewcommand{\b}{\bigg}
\newcommand{\Res}[2]{{\rm Res}(#1;#2)}
\newcommand{\To}{\quad\Rightarrow\quad}
\newcommand{\eps}{\epsilon}
\newcommand{\inner}[2]{\langle #1 , #2 \rangle}


\newcommand{\bt}[1]{\boldsymbol{#1}}
\newcommand{\mat}[1]{\textsf{\textbf{#1}}}
\newcommand{\I}{\boldsymbol{\mathcal{I}}}
\newcommand{\p}{\partial}
%
% Navn og tittel
%
\author{}
\title{Notes in FYS4130---Statistical physics}


\begin{document}

\section*{Laws of Thermodynamics}

\subsubsection*{First law}
The first law is a statement about conservation of energy. If we let $U$ denote the internal energy of a system, we state that we can decompose this energy change as \emph{heat}, $Q$ and \emph{work}, $W$.
$$\Delta U = Q \pm W.$$
Note that we use $\pm$ to denote the work, this is because we can define the work both \emph{on} the system or \emph{by} the system, both conventions are used regurarly and so care should be taken. The heat is usually defined as the heat \emph{entering} the system.

If the number of particles in a system is constant the instantaneous work done by the system will be 
$$\d W = P \d V,$$
so we have
$$\d U = \d Q - P \d V.$$

\subsection*{Second law}
The second law states that the entropy of an isolated system always will increase. Mathematically we can state it as
$$\Delta S \geq T\Delta Q,$$
at least for a constant temperature. For a reversible process, this becomes an equality
$$\Delta S = T \Delta Q \quad \mbox{(reversible process)} $$
This implies that a reversible process is a process that produces no entropy, it is thus a system continously changing through equilibria states and will therefore often be 
infinitely slow. A reversible process is thus ususally an abstraction, and not a real quality of a process. Processes can however, be close to reversible.

Combining the first and second law gives us the important inequality
$$T\Delta S \geq \Delta U + P\Delta V.$$
For an infinitesimal quantity, this will always be true
$$T\ \d S = \d U + P\ \d V.$$

\section*{Derivatives of state variables}
From the equation
$$\d U = T\ \d S - P\ \d V + \mu\ \d N,$$
we get the derivatives
\begin{align*}
\b(\frac{\p S}{\p U}\b)_{V,N}=\frac{1}{T}, \qquad \b(\frac{\p S}{\p V}\b)_{U,N}=\frac{P}{V}, \qquad \b(\frac{\p S}{\p N}\b)_{U,V}=-\frac{\mu}{T}.
\end{align*}

\clearpage

\section*{Maxwell relations}
From the first law we have
\beq \d U = T\ \d S - P\ \d V. \eeq \label{eq:maxwell1}
Should therefore express $U$ as a function of $S$ and $V$, $U = U(S,V)$, can then look at the total derivative of $U$
\beq \d U = \bigg(\frac{\p U}{\p S}\bigg)_V \d S + \bigg(\frac{\p U}{\p V}\bigg)_S \d V. \eeq \label{eq:maxwell2}
Comparing \ref{eq:maxwell1} and \ref{eq:maxwell2} gives
$$T = \bigg(\frac{\p U}{\p S}\bigg)_V, \qquad P = -\bigg(\frac{\p U}{\p V}\bigg)_S.$$

Generally
$$\frac{\p^2 U}{\p V \p S} = \frac{\p^2 U}{\p S \p V}.$$
So we get
$$\bigg(\frac{\p T}{\p V}\bigg)_S = -\bigg(\frac{\p P}{\p S}\bigg)_V.$$

\subsubsection*{General state variables}
Let us say we have three state variables, $X$, $Y$, $Z$, and we have an equation of state $Z(X,Y)$. We see that $Z$ is not a free variable, but is given by $X$ and $Y$. Of course, we could have said that $X$ is not the free variable, because it can be given by $X(Y,Z)$, i.e., the equation of state can be solved for any of the three state variables.

The total derivatives of all equation of states become
\begin{align*}
\d Z = \bigg(\frac{\p Z}{\p X}\bigg)_Y \d X + \bigg(\frac{\p Z}{\p Y}\bigg)_X \d Y, \\
\d X = \bigg(\frac{\p X}{\p Y}\bigg)_Z \d Y + \bigg(\frac{\p X}{\p Z}\bigg)_Y \d Z, \\
\d Y = \bigg(\frac{\p Y}{\p X}\bigg)_Z \d X + \bigg(\frac{\p Y}{\p Z}\bigg)_X \d Z.
\end{align*}
Inserting $\d Y$ into $\d X$ gives
$$\bigg[\bigg(\frac{\p X}{\p Y}\bigg)_Z\bigg(\frac{\p Y}{\p X}\bigg)_Z - 1\bigg]\d X  + \bigg[\bigg(\frac{\p X}{\p Y}\bigg)_Z\bigg(\frac{\p Y}{\p Z}\bigg)_X + \bigg(\frac{\p X}{\p Z}\bigg)_Y\bigg]\d Z = 0.$$
The differentials are of course independant, so we get
$$\bigg(\frac{\p X}{\p Y}\bigg)_Z = \bigg(\frac{\p Y}{\p X}\bigg)_Z^{-1},$$
and
$$\bigg(\frac{\p X}{\p Y}\bigg)_Z \bigg(\frac{\p Y}{\p Z}\bigg)_X \bigg(\frac{\p Z}{\p X}\bigg)_Y = -1.$$

We can now let another state variable be given by $X$ and $Y$
$$\d U = \bigg(\frac{\p U}{\p X}\bigg)_Y \d X + \bigg(\frac{\p U}{\p Y}\bigg)_X \d Y.$$
Dividing by $\d X$ and holding the equation constant at $Z$ gives
$$\b( \frac{\p U}{\p X} \b)_Z = \bigg(\frac{\p U}{\p X}\b)_Y + \b(\frac{\p U}{\p Y}\b)_X \b(\frac{\p Y}{\p X}\b)_Z$$

\subsection*{Specific Heat}
When we add a small amount of heat to a system, the temprature will rise by some small amount, this is the definition of the specific heat of that system. For a given state variable $X$ it is defined as
$$C_X = \lim_{\Delta T \to 0} \b(\frac{\Delta Q}{\Delta T}\b)_X = T \b(\frac{\p S}{\p T}\b)_X.$$

Starting at the first law we have
$$T \d S = \d U + P\ \d V,$$
inserting the complete derivative for $\d U$ gives
$$T \d S = \b(\frac{\p U}{\p T}\b)_V \d T + \b(\frac{\p U}{\p V}\b)_T \d V + P\ \d V,$$
If we look at the specific heat for a constant volume, we get
$$C_V = \b(\frac{\p U}{\p T}\b)_V.$$
If we instead keep the pressure constant, we get
$$C_P = C_V + \b[\b(\frac{\p U}{\p V}\b)_T + P\b]\ \b(\frac{\p V}{\p T}\b)_P.$$
This can also be formulated as
$$C_P = \b(\frac{\p H}{\p T}\b)_P.$$
Where $H$ is the \emph{enthalpy}:
$$H = U + PV.$$

\section*{Equation of State}
For most systems all state variables are not independent, but will depend on each other. If we can explicitly give a state variable in terms of the other state variables, that is an equation of state. An example could be the relation $$P(T,V).$$
 
The equation of state give important information about how matter will behave under different physical conditions and so is very material dependant.

\subsection*{Ideal gas law}
The ideal gas law is an equation of state
$$P = \frac{NkT}{V}, \qquad P = kT\rho.$$

\subsection*{Van der Waals Equation of State}
The ideal gas presupposes no interaction between the particles. The van der Waals equation of state tries to include some interaction in a basic and approximate manner
$$P = \frac{NkT}{V - Nb} - \frac{aN^2}{V^2}.$$
Here $a$ and $b$ are parameters describing the particles, $b$ reflects the particles having some volume and $a$ models the attraction between the particles.

\subsection*{Phase Transition}
Two phases are in thermodynamic equilibrium, meaning the total entropy
$$S(U,V,N) = S_1(U_1,V_1,N_1) + S_2(U_2,V_2,N_2),$$
is maximized. Taking the total derivative
\begin{align*}
\d S &= \b(\frac{\p S_1}{\p U_1}\b)_{V_1, N_1} \d U_1 + \b(\frac{\p S_1}{\p V_1}\b)_{U_1, N_1} \d V_1 + \b(\frac{\p S_1}{\p N_1}\b)_{U_1, V_1} \d N_1 \\
&\qquad + \b(\frac{\p S_2}{\p U_2}\b)_{V_2, N_2} \d U_2 + \b(\frac{\p S_2}{\p V_2}\b)_{U_2, N_2} \d V_2 + \b(\frac{\p S_2}{\p N_2}\b)_{U_2, V_2} \d N_2 \\
\end{align*}
Now, using the fact that $\d U_1 = -\d U_2$ etc, we can simplify this to
\begin{align*}
\d S &= \b[\b(\frac{\p S_1}{\p U_1}\b)_{V_1, N_1} - \b(\frac{\p S_2}{\p U_2}\b)_{V_2, N_2} \b] \d U_1 \\
&\qquad  + \b[\b(\frac{\p S_1}{\p V_1}\b)_{U_1, N_1} - \b(\frac{\p S_2}{\p V_2}\b)_{U_2, N_2}\b]\d V_1 \\
&\qquad\qquad+ \b[\b(\frac{\p S_1}{\p N_1}\b)_{U_1, V_1} - \b(\frac{\p S_2}{\p N_2}\b)_{U_2, V_2} \b]\d N_1 \\
\end{align*}

\clearpage

\section*{Ensambles}

In a microcanonical ensamble, the energy of the system is held constant. All microstates with the given energy are possible and of equal probability. Also known as a $NVE$-ensamble as the number of particles, volume and energy are constant.

From wikipedia:
In simple terms, the microcanonical ensemble is defined by assigning an equal probability to every microstate whose energy falls within a range centered at E. All other microstates are given a probability of zero. Since the probabilities must add up to 1, the probability P is the inverse of the number of microstates W within the range of energy,
$P = 1/W$,
The range of energy is then reduced in width until it is infinitesimally narrow, still centered at E. In the limit of this process, the microcanonical ensemble is obtained.

\clearpage

\section*{Phase space}

In general, a system of $N$ particles is described by the particles position and momenta in all dimensions, these evolve in time according to some dynamical equations of the system, so we have
$$q_i = q_i(t, q_0, p_0), \qquad p_i = p_i(t,q_0,p_0).$$
Where $(q_0, p_0)$ are the inital conditions of the system. For most systems, the evolution can only be found numerically. A problem then, is that the systems are often chaotic, and so the finite precision in the initial conditions limits our possible solutions.

A goal is to find the average of some quantity over a long time
$$\bar{A} = \frac{1}{T}\int_0^T A(q(t), p(t)) \ \d t.$$
And we will have to include states corresponding to all the parts of phase space visited, however, this requires knowledge of the phase trajectory, which we generally don't have.

A solution put forth by Gibbs is to instead look at phase space as filled by a liquid, the density in phase space is then given by $\rho = \rho(q,p)$ and the number of systems in a volume element $\Delta q \Delta p$ is the given by $\rho(q,p)\Delta q \Delta p$. The greater the density is in a region of phase space, the greater the probability is to find the system in that part of phase space. And so we can now average over the density of the entire phase space to find thermodynamic quantities.

In quantum mechanics, due to the Heisenberg uncertainty principle, a given particle state has the volume
$$\d\omega = \frac{\d^f q\  \d^f p}{(2\pi \hbar)^f}.$$

We of course choose the density of phase space in a manner that it is normalized in the sense that
$$\int \d \omega \rho(q,p) = 1.$$
Ensamble averages are then found from
$$\langle A \rangle = \int \rho(q,p) A(q,p) \ \d \omega.$$

It cannot be proven generally that the time-average, and the ensamble-average found from the density of phase space are equal, and we are now finding the ensamble-average through derivation, but obviously the time-average in experiment. For some systems they can be shown to be equal, but in general we have to simply invoke the ergodic hypothesis that state they are equal.

\subsection*{Liouville's Theorem}
To derive the behaviour of the density of phase space we can use the fact that the density of phase space is goverened by Hamilton's equations for each point. Also, no system can make discontinious jumps in the dynamical variables $q$ and $p$, and cannot start/stop exisiting, meaning the equation of continuity applies to the density
$$\frac{\p \rho}{\p t} + \nabla \cdot \vec{J} = 0.$$
Where $\vec{J}$ is the current of system points, which can be expressed as $\vec{J} = \rho\vec{V}$, where $\vec{V}$ is the velocity vector $\vec{V} = (\dot{\vec{q}}, \dot{\vec{p}})$. Writing out the nabla operator, we see that we can use Poisson brackets to compactily write the equation of continuity
$$\nabla \cdot \vec{J} = \sum_i \b( \frac{\p p}{\p q_i}\frac{\p H}{\p p_i} - \frac{\p H}{\p q_i}\frac{\p \rho}{\p p_i}\b).$$
\subsubsection*{Equation of continuity}
$$\frac{\p p}{\p t} + \{\rho, H\} = 0.$$

Now, the phase space density is generally a time-dependant field, $\rho(q,p,t)$ and so the total derivative is
$$\frac{\d }{\d t} \rho = \frac{\p \rho}{\p t} + \{\rho, H\},$$
which we see from the equation of continuity is zero. This is Liouville's theorem. Stated in words it says that the local density of phase space remains constant as seen by an observer moving with a system point.

If we then follow an initial volume element with a constant density, it will remain at a constant volume, but generally change shape as time evolves. 

An important consequence of Liouville's theorem is that if $\rho$ is initially constant throughout phase space, meaning all microstates are equally likely. It will remain so as time evolves.

Ergodicity means that the system will eventually visit all of the phase space that is covered by the ensemble, and for ergodic system Liouvilles theorem implies that every microstate is equally likely to be visited with time.



\subsection*{Poisson Bracket}
In canonical coordinates, the Poisson bracket takes the form
$$\{f,g\} = \sum_{i=1}^N \b(\frac{\p f}{\p q_i}\frac{\p g}{\p p_i} - \frac{\p f}{\p p_i}\frac{\p g}{\p q_i}\b).$$
It has the same role as the commutator has in quantum mechanics.

Any functions, $f, g, h$ of phase space and time follows
\begin{itemize}
	\item Anticommutativity: $\{f, g\} = - \{g, f\}$
	\item Distributivity: $\{f+g, h\} = \{f, h\} + \{g, h\}$
	\item Product rule: $\{fg, h\} = \{f, h\}g + f\{g, h\}$
	\item If $k$ is time-dependant, but constant over phase space, then $\{f,k\} = 0$.
\end{itemize}
For cannonical coordinates
\begin{itemize}
	\item $\{q_i, q_j\} = 0$
	\item $\{p_i, p_j\} = 0$
	\item $\{p_i, q_j\} = \delta_{ij}$
\end{itemize}

Using the Poisson bracket, we can write Hamilton's equations as
$$\dot{q} = \{q, H(q,p)\}, \qquad \dot{p} = \{p, H(q,p)\}.$$

More generally, for \emph{any} function that is not explicitly dependant on time, but only $q$ and $p$
$$\frac{d}{dt}f(q,p) = \{f, H\},$$
if there is an explicit time-dependance we have
$$\frac{d}{dt}f(q,p,t) = \{f, H\} + \frac{\p f}{\p t}.$$


\clearpage

\section*{Other}

\subsubsection*{Stirling's approximation}
Coarse
$$\log n! = n \log n - n,$$
Fine
$$n! = \sqrt{2\pi n} n^n e^{-n}.$$

\section*{Constants}

\begin{center}
\begin{tabular}{|c|c|}	
\hline
Boltzmann's constant & $k = 1.381\times10^{-23} \mbox{JK}^{-1}$ \\ \hline 
Avogadro's Number & $N_{\rm A} = 6.023\times10^{23} \mbox{mol}^{-1}$ \\ \hline 
$k\cdot N_{\rm A}$ & $R = 8.314 \mbox{JK}^{-1}\mbox{mol}^{-1}$ \\ \hline
\end{tabular}
\end{center}


\end{document}



