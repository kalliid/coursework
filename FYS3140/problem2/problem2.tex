\documentclass[a4paper, 11pt, titlepage, english]{article}

\usepackage{babel}
\usepackage[latin1]{inputenc}
\usepackage[T1]{fontenc, url}
\usepackage{textcomp}
\usepackage{amsmath, amssymb}
\usepackage{amsbsy, amsfonts}
\usepackage{graphicx, color}
\usepackage{parskip}
\usepackage{framed}
\usepackage{amsmath}
\usepackage{xcolor}
\usepackage{multicol}
\usepackage{url}
\usepackage{flafter}


\usepackage{geometry}
\geometry{headheight=0.01mm}
\geometry{top=24mm, bottom=30mm, left=39mm, right=39mm}

%
% Parametere for inkludering av kode fra fil
%
\usepackage{listings}
\lstset{language=python}
\lstset{basicstyle=\ttfamily\small}
\lstset{frame=single}
\lstset{keywordstyle=\color{red}\bfseries}
\lstset{commentstyle=\itshape\color{blue}}
\lstset{showspaces=false}
\lstset{showstringspaces=false}
\lstset{showtabs=false}
\lstset{breaklines}

%
% Definering av egne kommandoer og miljøer
%
\newcommand{\dd}[1]{\ \text{d}#1}
\newcommand{\f}[2]{\frac{#1}{#2}} 
\newcommand{\beq}{\begin{equation*}}
\newcommand{\eeq}{\end{equation*}}
\newcommand{\bra}[1]{\langle #1|}
\newcommand{\ket}[1]{|#1 \rangle}
\newcommand{\braket}[2]{\langle #1 | #2 \rangle}
\newcommand{\braup}[1]{\langle #1 \left|\uparrow\rangle\right.}
\newcommand{\bradown}[1]{\langle #1 \left|\downarrow\rangle\right.}
\newcommand{\av}[1]{\left| #1 \right|}
\newcommand{\op}[1]{\hat{#1}}
\newcommand{\braopket}[3]{\langle #1 | {#2} | #3 \rangle}
\newcommand{\ketbra}[2]{\ket{#1}\bra{#2}}
\newcommand{\pp}[1]{\frac{\partial}{\partial #1}}
\newcommand{\ppn}[1]{\frac{\partial^2}{\partial #1^2}}
\newcommand{\up}{\left|\uparrow\rangle\right.}
\newcommand{\upup}{\left|\uparrow\uparrow\rangle\right.}
\newcommand{\down}{\left|\downarrow\rangle\right.}
\newcommand{\downdown}{\left|\downarrow\downarrow\rangle\right.}
\newcommand{\updown}{\left|\uparrow\downarrow\rangle\right.}
\newcommand{\downup}{\left|\downarrow\uparrow\rangle\right.}
\newcommand{\bupup}{\left.\langle\uparrow\uparrow\right|}
\newcommand{\bdowndown}{\left.\langle\downarrow\downarrow\right|}
\newcommand{\bupdown}{\left.\langle\uparrow\downarrow\right|}
\newcommand{\bdownup}{\left.\langle\downarrow\uparrow\right|}
\renewcommand{\d}{{\rm d}}

\makeatletter
\renewcommand*\env@matrix[1][*\c@MaxMatrixCols c]{%
  \hskip -\arraycolsep
  \let\@ifnextchar\new@ifnextchar
  \array{#1}}
\makeatother

%
% Navn og tittel
%
\author{Jonas van den Brink}
\title{Problem set 2 \\ FYS3140}


\begin{document}
\maketitle
% \newpage\null\thispagestyle{empty}\newpage

% \setcounter{page}{1} 

\section*{Problem 2.1}
Functions of complex variables can be written in the general form $f(z)=u(x,y) + i v(x,y)$. We will now find $u(x,y)$ and $v(x,t)$ for two different functions.
\subsection*{a)}
The function 
$$\frac{-i+2z}{2+iz},$$
we start by writing the complex variable as $z=x+iy$, and then simplify the fraction.
$$\frac{-i+2z}{2+iz} = \frac{2x + (2y-1)i}{(2-y) + xi} = \frac{(2x + (2y-1)i)((2-y) + xi)}{(2-y)^2 + x^2},$$
multplying out gives
$$\frac{2x(2-y) + 2x^2i + (2y-1)(2-y)i - x(2y-1)i}{4 - 2y + y^2 + x^2},$$
which can be written as
$$\bigg( \frac{4x - 2xy}{x^2 + y^2 -2y + 4}\bigg) + \bigg(\frac{2x^2 -2y^2-2xy+5y+x-2}{x^2 +y^2-2y+4}\bigg)i.$$

\subsection*{b)}
The function $e^{iz}$. We start by writing the complex variable as $z = x + iy$, and then use Eulers formula:
$$ e^{iz} = e^{i(x+iy)} = e^{-y + ix} = e^{-y}e^{ix} = e^{-y}\bigg(\cos x + i \sin x\bigg), $$
giving
$$u(x,y) = e^{-y}\cos x, \qquad v(x,y) = e^{-y}\sin x.$$

\clearpage 

\section*{Problem 2.2 (Derivatives)}
We will now use the definition of the complex derivative
$$\frac{\d f}{\d z} = \lim_{\Delta z \to 0} \frac{\Delta f}{\Delta z},$$
to show that the product rule holds for the functions of complex variables.

We start by using the definition of the derivative on $[f(z)g(z)]$:
$$\frac{\d}{\d z}[f(z)g(z)] = \lim_{\Delta z \to 0} \frac{\Delta [ f(z)g(z) ]}{\Delta z},$$
we now expand $\Delta [ f(z)g(z) ]$,
$$\Delta [ f(z)g(z) ] = g(z) \Delta f + f(z)\Delta g,$$
giving us
$$\frac{\d}{\d z}[f(z)g(z)] = \lim_{\Delta z \to 0} \frac{g(z)\Delta f + f(z) \Delta g}{\Delta z},$$
which can be further rewritten to
$$\frac{\d}{\d z}[f(z)g(z)] = g(z) \lim_{\Delta z \to 0} \frac{\Delta f}{\Delta z} + f(z) \lim_{\Delta z \to 0} \frac{\Delta g}{\Delta z},$$
or
$$\frac{\d}{\d z}[f(z)g(z)] = g(z)\frac{\d f}{\d z} + f(z)\frac{\d g}{\d z}, \quad {\rm q.e.d.}$$

\section*{Problem 2.3 (Cauchy-Riemann conditions)}
We will now derive the Cauchy-Riemann conditions in polar coordinates.

We have a function of a complex variable $z$, where $z= re^{i\theta}$, by partial differentiation we have
$$\frac{\partial f}{\partial r} = \frac{\d f}{\d z} \frac{\partial z}{\partial r} = \frac{\d f}{\d z}\cdot e^{i\theta},$$
and
$$\frac{\partial f}{\partial \theta} = \frac{\d f}{\d z} \frac{\partial z}{\partial \theta} = \frac{\d f}{\d z}\cdot ire^{i\theta}.$$
Since $f = u(r,\theta) + iv(r,\theta)$, we also have
$$\frac{\partial f}{\partial r} = \frac{\partial u}{\partial r} + i\frac{\partial v}{\partial r}, \quad {\rm and} \quad \frac{\partial f}{\partial \theta} = \frac{\partial u}{\partial \theta} + i\frac{\partial v}{\partial \theta}.$$

By combining these equations we find
$$\frac{\d f}{\d z}\cdot e^{i\theta} = \frac{\partial u}{\partial r} + i\frac{\partial v}{\partial r}, \qquad {\rm and}\qquad \frac{\d f}{\d z}\cdot ire^{i\theta} = \frac{\partial u}{\partial \theta} + i\frac{\partial v}{\partial \theta}.$$

We demand that the derivative $\d f/\d z$ is unique, meaning we can combine the equations to eliminate it, inserting the first into the second gives:
$$\bigg(\frac{\partial u}{\partial r} + i\frac{\partial v}{\partial r}\bigg)ir = \frac{\partial u}{\partial \theta} + i\frac{\partial v}{\partial \theta}.$$
Setting the real and imaginary parts equal gives the Cauchy-Riemann conditions in polar coordinates
$$r\frac{\partial u}{\partial r} = \frac{\partial v}{\partial \theta}, \qquad r\frac{\partial v}{\partial r} = -\frac{\partial u}{\partial \theta}.$$

\clearpage

\section*{Problem 2.4 (Harmonic Functions)}
\subsection*{a)}
We will show that the following function is harmonic,
$$u(x,y) = \frac{y}{(1-x)^2 + y^2},$$
by definition, this means that it is a solution to Laplace's equation:
$$\nabla^2 u = \frac{\partial^2 u}{\partial x^2} + \frac{\partial^2 u}{\partial y^2} = 0.$$

To show this, we calculate the partial derivatives seperately. We also introduce the shorthand notation
$$\kappa \equiv \big[(1-x)^2 + y^2\big] = x^2 - 2x + y^2 + 1, \qquad u = y\kappa^{-1}.$$
The explicit derivatives with respect to $x$ gives:
$$\frac{\partial u}{\partial x} = 2y(1-x)\kappa^{-2}, \qquad \frac{\partial^2 u}{\partial x^2} = -2y\kappa^{-2} + 8y(1-x)^2\kappa^{-3}.$$
And for $y$ we get
$$\frac{\partial u}{\partial y} = \kappa^{-1} - 2y^2\kappa^{-2}, \qquad \frac{\partial^2 u}{\partial y^2} = -2y\kappa^{-2}-4y\kappa^{-2} + 8y^3\kappa^{-3}.$$
Combining the expressions now gives
$$\nabla^2 u = -8y\kappa^{-2} + 8y(1-x)^2\kappa^{-3} + 8y^3\kappa^{-3},$$
which we can rewrite to
$$\nabla^2 u = -8y\kappa^{-2} + 8y\bigg[(1-x)^2 + y^2\bigg]\kappa^{-3} = -8y\kappa^{-2} + 8y\kappa^{-2} = 0.$$
So we see that the Laplacian does indeed vanish, and so $u$ is, by definition, a harmonic function.

\clearpage

\subsection*{b)}
We will now use the Cauchy-Riemann equations to find the harmonic conjugate $v(x,y)$. 

The first conditions demands
$$ \frac{\partial v}{\partial y} = \frac{\partial u}{\partial x} = 2y(1-x)\kappa^{-2},$$
which gives
$$v(x,y) = \int 2y(1-x)\kappa^{-2} {\rm\ d}y =  (x-1)\kappa^{-1} + g(x).$$

The second condition demands
$$\frac{\partial v}{\partial x} = - \frac{\partial u}{\partial y} = 2y^2\kappa^{-2} - \kappa^{-1} = \bigg[y^2 - (1-x)^2\bigg]\kappa^{-2},$$
but we can also calculate $\partial v/\partial x$ from the expression we found for $v$ earlier:
$$\frac{\partial v}{\partial x} = \kappa^{-1} - (1-x)^2\kappa^{-2} = \bigg[y^2 - (1-x)^2\bigg]\kappa^{-2},$$
so we see that $g(x) = 0$, and we have
$$v(x,y) = (x-1)\kappa^{-1}.$$



\subsection*{c)}
Just to check that $v$ does indeed satisfy Laplace's equation, we calculate $\partial^2 v/\partial x^2$ and $\partial^2 v/\partial y^2$. We already know the first derivatives, from the Cauchy-Riemann conditions, which gives
$$\frac{\partial^2 v}{\partial x^2} = \frac{\partial}{\partial x}\bigg(2y^2\kappa^{-2}-\kappa^{-1}\bigg) = 8y^2(1-x)\kappa^{-3} - 2(1-x)\kappa^{-2},$$
and
$$\frac{\partial^2 v}{\partial y^2} = \frac{\partial}{\partial y}\bigg(2y(1-x)\kappa^{-2}\bigg) = 2(1-x)\kappa^{-2} - 8y^2(1-x)\kappa^{-3}.$$
Summing these gives
$$\nabla^2 v(x,y) = \frac{\partial^2 v}{\partial x^2} + \frac{\partial^2 v}{\partial y^2} = 0,$$
so $v(x,y)$ does indeed satisfy Laplace's equation.


\clearpage

\section*{Extra problems}
\subsection*{2.17.25}
\textbf{(a)}---Show that $\overline{\cos(z)} = \cos \overline{z}$.

This is easily shown by writing $\cos$ in term of the exponential functions:
$$\overline{\cos(z)} = \overline{\frac{e^{iz} + e^{-iz}}{2}} = \frac{\overline{e^{iz}} + \overline{e^{-iz}}}{2} = \frac{e^{-i\overline{z}}+ e^{i\overline{z}}}{2} = \cos \overline{z}.$$

\vspace{0.5cm}

\textbf{(b)}---Is $\overline{\sin z} = \sin \overline{z}$?

Let us try the same method as in the previous case:
$$\overline{\sin(z)} = \overline{\bigg(\frac{e^{iz} - e^{-iz}}{2i}\bigg)} = \frac{\overline{e^{iz}} - \overline{e^{-iz}}}{\overline{2i}} = \frac{e^{-i\overline{z}}- e^{i\overline{z}}}{-2i} = \frac{e^{i\overline{z}}-e^{-i\overline{z}}}{2i} = \sin \overline{z}.$$
So we see that the answer is yes---$\overline{\sin z} = \sin \overline{z}$ for all $z\in\mathbb{C}$.

\vspace{1cm}

\textbf{(c)}---If $f(z) = 1 + iz$, is $\overline{f(z)} = f(\overline{z})$?

This is easy to test:
$$\overline{f(z)} = \overline{1 + iz} = 1 + \overline{iz} = 1 - i\overline{z} \neq f(\overline{z}),$$
We see that the two are not equivalent.

\vspace{1cm}

\textbf{(d)}---If the power series of $f(z)$ has only real coefficients, show $\overline{f(z)} = f(\overline{z})$.

We use the fact that the coefficients are real, i.e., $\overline{a_k} = a_k$, to show:
$$\overline{f(z)} = \overline{\sum_k a_k z^k} = \sum_k a_k \overline{z}^k = f(\overline{z}).$$

\vspace{0.5cm}

\textbf{(e)}---Using the result of $(d)$, verify that $i[\sinh(1+i) - \sinh(1-i)]$ is real.

If we write $\sinh$ in terms of the exponential functions:
$$\sinh z = \frac{e^{z} - e^{-z}}{2},$$
we see that the power series of $\sinh$ only has real coefficients, meaning 
$$\overline{\sinh(z)} = \sinh(\overline{z}),$$
and we have
$$i[\sinh(1+i) - \sinh(1-i)] = i[\sinh(1+i) - \overline{\sinh(1+i)}] = -\Im[\sinh(1+i)],$$
where $\Im$ denotes the imaginary part of the argument, which is purely real, as was to be shown.

\clearpage

\subsection*{2.17.28}
Evaluate the absolute square of a complex number, assume $a$ and $b$ are real.
$$|c|^2 = \bigg|\frac{(a+bi)^2e^b - (a-bi)^2e^{-b}}{4abie^{-ia}}\bigg|^2,$$
using the fact that $|c|^2 = \overline{c}\cdot c$, gives
\begin{align*}
|c|^2 &= \overline{\bigg(\frac{(a+bi)^2e^b - (a-bi)^2e^{-b}}{4abie^{-ia}}\bigg)}\bigg(\frac{(a+bi)^2e^b - (a-bi)^2e^{-b}}{4abie^{-ia}}\bigg) \\[0.2cm]
&= \bigg(\frac{(a-bi)^2e^b - (a+bi)^2e^{-b}}{-4abie^{ia}}\bigg)\bigg(\frac{(a+bi)^2e^b - (a-bi)^2e^{-b}}{4abie^{-ia}} \bigg) \\[0.2cm]
&= \frac{(a+bi)^2(a-bi)^2\big(e^{2b}+e^{-2b}\big) - (a+bi)^4 - (a-bi)^4}{16a^2b^2} \\[0.2cm]
&= \frac{\big(a^2 + b^2\big)^2}{8a^2b^2}\cosh(2b) - \frac{(a+bi)^4 + (a-bi)^4}{16a^2b^2}.
\end{align*}

\vspace{1cm}

\subsection*{2.17.32}
We will show that
$$\sum_{n=0}^\infty \frac{(1+i\pi)^n}{n!} = -e.$$

We recognize the sum as the expansion of $e^z$, so we have
$$\sum_{n=0}^\infty \frac{(1+i\pi)^n}{n!} = e^{1+i\pi} = ee^{i\pi} = e(-1) = -e, \qquad {\rm q.e.d.}$$


\end{document}

