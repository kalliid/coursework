\documentclass[a4paper, 11pt, titlepage, english]{article}

\usepackage{babel}
\usepackage[utf8]{inputenc}
\usepackage[T1]{fontenc, url}
\usepackage{textcomp}
\usepackage{amsmath, amssymb}
\usepackage{amsbsy, amsfonts}
\usepackage{graphicx, color}
\usepackage{parskip}
\usepackage{framed}
\usepackage{amsmath}
\usepackage{xcolor}
\usepackage{multicol}
\usepackage{url}
\usepackage{flafter}


\usepackage{geometry}
\geometry{headheight=0.01mm}
\geometry{top=24mm, bottom=30mm, left=39mm, right=39mm}

%
% Parametere for inkludering av kode fra fil
%
\usepackage{listings}
\lstset{language=python}
\lstset{basicstyle=\ttfamily\small}
\lstset{frame=single}
\lstset{keywordstyle=\color{red}\bfseries}
\lstset{commentstyle=\itshape\color{blue}}
\lstset{showspaces=false}
\lstset{showstringspaces=false}
\lstset{showtabs=false}
\lstset{breaklines}

%
% Definering av egne kommandoer og miljøer
%
\newcommand{\dd}[1]{\ \text{d}#1}
\newcommand{\f}[2]{\frac{#1}{#2}} 
\newcommand{\beq}{\begin{equation*}}
\newcommand{\eeq}{\end{equation*}}
\newcommand{\bra}[1]{\langle #1|}
\newcommand{\ket}[1]{|#1 \rangle}
\newcommand{\braket}[2]{\langle #1 | #2 \rangle}
\newcommand{\braup}[1]{\langle #1 \left|\uparrow\rangle\right.}
\newcommand{\bradown}[1]{\langle #1 \left|\downarrow\rangle\right.}
\newcommand{\av}[1]{\left| #1 \right|}
\newcommand{\op}[1]{\hat{#1}}
\newcommand{\braopket}[3]{\langle #1 | {#2} | #3 \rangle}
\newcommand{\ketbra}[2]{\ket{#1}\bra{#2}}
\newcommand{\pp}[1]{\frac{\partial}{\partial #1}}
\newcommand{\ppn}[1]{\frac{\partial^2}{\partial #1^2}}
\newcommand{\up}{\left|\uparrow\rangle\right.}
\newcommand{\upup}{\left|\uparrow\uparrow\rangle\right.}
\newcommand{\down}{\left|\downarrow\rangle\right.}
\newcommand{\downdown}{\left|\downarrow\downarrow\rangle\right.}
\newcommand{\updown}{\left|\uparrow\downarrow\rangle\right.}
\newcommand{\downup}{\left|\downarrow\uparrow\rangle\right.}
\newcommand{\bupup}{\left.\langle\uparrow\uparrow\right|}
\newcommand{\bdowndown}{\left.\langle\downarrow\downarrow\right|}
\newcommand{\bupdown}{\left.\langle\uparrow\downarrow\right|}
\newcommand{\bdownup}{\left.\langle\downarrow\uparrow\right|}
\renewcommand{\d}{{\rm d}}

\makeatletter
\renewcommand*\env@matrix[1][*\c@MaxMatrixCols c]{%
  \hskip -\arraycolsep
  \let\@ifnextchar\new@ifnextchar
  \array{#1}}
\makeatother

%
% Navn og tittel
%
\author{Jonas van den Brink}
\title{Problem set 3 \\ FYS3140}


\begin{document}
\maketitle
\newpage\null\thispagestyle{empty}\newpage

\setcounter{page}{1} 

\section*{Problem 3.1 (Cauchy's Theorem and integral formula)}
\subsection*{a)}
We will evaulate the integral
$$\oint_\Gamma \frac{\sin z}{2z - \pi} \ \d z,$$
where $\Gamma$ is the circle $|z|=3$, as $\sin(z)$ is an entire function, we know that it is analytic on and inside $\Gamma$ and we can apply Cauchy's integral formula, giving
$$\oint_\Gamma \frac{\sin z}{2z - \pi} \ \d z = \frac{1}{2}\oint_\Gamma \frac{\sin z}{z - \pi/2} = \pi i \sin\bigg(\frac{\pi}{2}\bigg) = \pi i.$$

\subsection*{b)}
We will evaulate the integral
$$\oint_\Gamma \frac{\sin z}{2z - \pi} \ \d z,$$
where $\Gamma$ is the circle $|z|=1$, as the integrand has no poles inside $\Gamma$, Cauchy's theorem tells us that
$$\oint_\Gamma \frac{\sin z}{2z - \pi} \ \d z = 0.$$

\subsection*{c)}
We will evaulate the integral
$$\oint_\Gamma \frac{\sin z}{6z - \pi} \ \d z,$$
where $\Gamma$ is the circle $|z|=1$, as $\sin(z)$ is an entire function, we know that it is analytic on and inside $\Gamma$ and we can apply Cauchy's integral formula, giving
$$\oint_\Gamma \frac{\sin z}{6z - \pi} \ \d z = \frac{1}{6}\oint_\Gamma \frac{\sin z}{z - \pi/6} = \frac{\pi}{3} i \sin\bigg(\frac{\pi}{6}\bigg) = \frac{\pi}{6}i.$$

\subsection*{d)}
We will evaulate the integral
$$\oint_\Gamma \frac{e^{2z}}{z-\ln 2} \ \d z,$$
where $\Gamma$ is the square with vertices $\pm 2, \pm 2i$. As the exponential function is entire, we know that it is analytic on and inside $\Gamma$ and we can apply Cauchy's integral formula, giving
$$\oint_\Gamma \frac{e^{2z}}{z-\ln 2} \ \d z = 2\pi i e^{2\ln 2} = 8\pi i.$$

\clearpage

\section*{Problem 3.2 (Generalized Cauchy integral formula)}
We will now derive the generalized Cauchy integral formula,
$$f^{(n)}(z) = \frac{n!}{2\pi i} \oint_\Gamma \frac{f(\omega)}{(\omega-z)^{n+1}} \ \d \omega.$$

We start from the standard Cauchy integral formula
$$f(z) = \frac{1}{2\pi i} \oint_\Gamma \frac{f(\omega)}{(\omega-z)} \ \d \omega,$$
and we derivate with respect to $z$, as the integration is with respect to the introduced variable $\omega$, the derivative can be interchanged with the integration. Note also that as we demand that $f$ be analytic on and inside $\Gamma$, the derivatives of $f$ exist and are also analytic on and inside $\Gamma$. 

The first derivative then becomes
$$f'(z) = \frac{1}{2\pi i} \oint_\Gamma \frac{f(\omega)}{(\omega-z)^2} \ \d \omega,$$
derivativing again gives
$$f''(z) = \frac{2}{2\pi i} \oint_\Gamma \frac{f(\omega)}{(\omega-z)^3} \ \d \omega,$$
and so on. Through induction we easily achieve the generalized formula
$$f^{(n)}(z) = \frac{n!}{2\pi i} \oint_\Gamma \frac{f(\omega)}{(\omega-z)^{n+1}} \ \d \omega, \qquad (n=1,2,3,\ldots).$$

We will now use this formula to evaluate the integral
$$\oint_\Gamma \frac{\sin 2z}{(6z-\pi)^3},$$
where $\Gamma$ is the circle $|z| = 2$. Once again, we know that the sine-function is entire, and so Cauchy's generalized integral formula can be used, giving
\begin{align*}
\oint_\Gamma \frac{\sin 2z}{(6z-\pi)^3} &= \frac{1}{6^3}\oint_\Gamma \frac{\sin 2z}{(z-\pi/6)^3} \\[0.2cm]
&= \frac{1}{6^3}\frac{2\pi i}{2!}(-4)\sin\bigg(2\cdot\frac{\pi}{6}\bigg) \\
&= -\frac{\sqrt{3}}{108}i.
\end{align*}

\clearpage

\section*{Problem 3.3 (Laurent series)}
We will find the Laurent series about the origin of the function
$$f(z) = \frac{z-1}{z^2(z-2)},$$
in different domains.

\subsection*{a)}
In the punctured disk $0 < |z| < 2$. We first rewrite the function to the form
$$f(z) = \frac{z-1}{z^2(z-2)} = \frac{1-z}{2z^2}\cdot\frac{1}{1-\frac{z}{2}},$$
as $|z|<2$ in the disc, we see that $|z/2|<1$, recognizing the last fraction as the geometric series we get 
$$f(z)  = \frac{1-z}{2z^2}\sum_{k=0}^\infty \frac{z^k}{2^k} = \sum_{k=0}^\infty \bigg(\frac{z^{k-2}}{2^{k+1}} - \frac{z^{k-1}}{2^{k+1}}\bigg),$$
writing out the first few terms gives
$$f(z) = \frac{1}{2z^2} - \frac{1}{4z} - \sum_{k=0}^\infty \frac{z^k}{2^{k+3}}.$$

\subsection*{b)}
For $|z| > 2$. We start by rewriting $f(z)$ to the form:
$$f(z) = \frac{z-1}{z^3} \frac{1}{1-\frac{2}{z}},$$
as $|z|>2$, we see that $|2/z|<1$ and again we recognize the fraction as the geometric series, giving
$$f(z) =  \frac{z-1}{z^3} \sum_{k=0}^\infty \frac{2^k}{z^k} = \sum_{k=0}^\infty \bigg(\frac{2^k}{z^{k+2}} - \frac{2^k}{z^{k+3}}\bigg).$$

\subsection*{c)}
The residue of $f(z)$ at the origin we have already found in problem 3.3a, as the Laurent series converges inside the punctured disk $0 < |z| < 2$, the residue at the singular point $z_0 = 0$ is simply the coefficient $b_1$, meaning
$${\rm Res}(f;0) = -\frac{1}{4}.$$

\clearpage

\section*{Problem 3.4 (Singularities)}
We will now classify the singularities of some functions.
\subsection*{a)}
The function $f(z) = {\sin z}/{3z}$, at the origin, $z_0 = 0$. As the function has a finite limit as $z$ approaches the origin:
$$\lim_{z\to 0} \frac{\sin z}{3z} = \frac{1}{3},$$
the singularity is removable.

\subsection*{b)}
The function $f(z) = {\cos z}/{z^4}$, at the origin, $z_0 = 0$. As the cosine-function is non-zero at the origin, we see that this singularity is a fourth order pole.

\subsection*{c)}
The function
$$f(z) = \frac{z^3-1}{(z-1)^3},$$
at the point, $z_0 = 1$. We look at the function $g(z) \equiv 1/ f(z)$, as $f$ will have a pole of the same order as $g$ has order of zero in $z_0$, we see that
$$g(z) = \frac{(z-1)^3}{z^3 - 1}.$$
Using L'Hôpital's rule, we find
\begin{align*}
\lim_{z\to1} g(z) &= 0, \\
\lim_{z\to1} g'(z) &= 0, \\
\lim_{z\to1} g''(z) &= 2/3.
\end{align*}
As $g$ has a zero of second order at $z_0$, we know that $f(z)$ has a pole of second order here.

\subsection*{d)}
The function $f(z) = e^z/(z-1)$ at the point $z_0 = 1$. As $e^z$ is analytic and non-zero at $z_0$, we see that $f$ has a pole of first order at $z_0$.




\end{document}

