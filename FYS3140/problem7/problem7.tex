\documentclass[a4paper, 11pt, titlepage, english]{article}

\usepackage{babel}
\usepackage[utf8]{inputenc}
\usepackage[T1]{fontenc, url}
\usepackage{textcomp}
\usepackage{amsmath, amssymb}
\usepackage{amsbsy, amsfonts}
\usepackage{graphicx, color}
\usepackage{parskip}
\usepackage{framed}
\usepackage{amsmath}
\usepackage{xcolor}
\usepackage{multicol}
\usepackage{url}
\usepackage{flafter}


\usepackage{geometry}
\geometry{headheight=0.01mm}
\geometry{top=24mm, bottom=30mm, left=39mm, right=39mm}

%
% Parametere for inkludering av kode fra fil
%
\usepackage{listings}
\lstset{language=python}
\lstset{basicstyle=\ttfamily\small}
\lstset{frame=single}
\lstset{keywordstyle=\color{red}\bfseries}
\lstset{commentstyle=\itshape\color{blue}}
\lstset{showspaces=false}
\lstset{showstringspaces=false}
\lstset{showtabs=false}
\lstset{breaklines}

%
% Definering av egne kommandoer og miljøer
%
\newcommand{\dd}[1]{\ \text{d}#1}
\newcommand{\f}[2]{\frac{#1}{#2}} 
\newcommand{\beq}{\begin{equation*}}
\newcommand{\eeq}{\end{equation*}}
\newcommand{\bra}[1]{\langle #1|}
\newcommand{\ket}[1]{|#1 \rangle}
\newcommand{\braket}[2]{\langle #1 | #2 \rangle}
\newcommand{\braup}[1]{\langle #1 \left|\uparrow\rangle\right.}
\newcommand{\bradown}[1]{\langle #1 \left|\downarrow\rangle\right.}
\newcommand{\av}[1]{\left| #1 \right|}
\newcommand{\op}[1]{\hat{#1}}
\newcommand{\braopket}[3]{\langle #1 | {#2} | #3 \rangle}
\newcommand{\ketbra}[2]{\ket{#1}\bra{#2}}
\newcommand{\pp}[1]{\frac{\partial}{\partial #1}}
\newcommand{\ppn}[1]{\frac{\partial^2}{\partial #1^2}}
\newcommand{\up}{\left|\uparrow\rangle\right.}
\newcommand{\upup}{\left|\uparrow\uparrow\rangle\right.}
\newcommand{\down}{\left|\downarrow\rangle\right.}
\newcommand{\downdown}{\left|\downarrow\downarrow\rangle\right.}
\newcommand{\updown}{\left|\uparrow\downarrow\rangle\right.}
\newcommand{\downup}{\left|\downarrow\uparrow\rangle\right.}
\newcommand{\bupup}{\left.\langle\uparrow\uparrow\right|}
\newcommand{\bdowndown}{\left.\langle\downarrow\downarrow\right|}
\newcommand{\bupdown}{\left.\langle\uparrow\downarrow\right|}
\newcommand{\bdownup}{\left.\langle\downarrow\uparrow\right|}
\renewcommand{\d}{{\rm d}}
\newcommand{\Res}[2]{{\rm Res}(#1;#2)}
\newcommand{\To}{\quad\Rightarrow\quad}

\makeatletter
\renewcommand*\env@matrix[1][*\c@MaxMatrixCols c]{%
  \hskip -\arraycolsep
  \let\@ifnextchar\new@ifnextchar
  \array{#1}}
\makeatother



%
% Navn og tittel
%
\author{Jonas van den Brink}
\title{Problem set 7 \\ FYS3140}


\begin{document}
\maketitle
% \newpage\null\thispagestyle{empty}\newpage
% 
% \setcounter{page}{1} 

\section*{Boas 8.12.16}
The DE:
$$x^2y'' - 2xy' + 2y = x\ln x,$$
has the homogenous solution $y_h(x) = c_1 x + c_2 x^2$. 

We will find a particular solution of the DE by variation of parameters, we start by assuming the solution can be written on the form
$$y_p(x) = c(x)y_1(x) = xc(x).$$
the derivatives of $y_p$ are then
$$ y_p' = xc' + c, \qquad y_p'' = xc'' + 2c'.$$
Insertion into the original DE will then gives an equation for $c'(x)$:
$$x^3c'' + 2x^2c' -2x^2c' -2xc + 2xc = x \ln x \To x^3c''(x) = x\ln x.$$
So we have the equation
$$\frac{\d^2 c}{\d x^2} = \frac{\ln x}{x^2},$$
meaning we can find $c(x)$ by integrating twice, note that we can discard the integration constants, as we are looking for a particular solution,
$$\frac{\d c}{\d x} = \int \frac{\ln x}{x^2} \ \d x = -\frac{\ln x}{x} + \int \frac{1}{x^2} \ \d x = -\frac{\ln x + 1}{x},$$
giving
$$c(x) = - \int \frac{\ln x}{x} \ \d x -\int\frac{1}{x}\ \d x = -\frac{1}{2}\ln^2 x - \ln x.$$
So the particular solution is
$$y_p(x) = xc(x) = -\frac{1}{2}x\ln^2 x - x\ln x,$$
and the full solution of the DE is then
$$y(x) = c_1 x + c_2 x^2  -\frac{1}{2}x\ln^2 x - x\ln x.$$

\clearpage

\section*{Boas 8.12.18}
The DE:
$$(x^2+1)y'' - 2xy' + 2y = (x^2+1)^2,$$
has the homogenous solution $y_h(x) = c_1 x + c_2 (1-x^2)$. 

We will find a particular solution of the DE by variation of parameters, we start by assuming the solution can be written on the form
$$y_p(x) = c(x)y_1(x) = xc(x).$$
the derivatives of $y_p$ are then
$$ y_p' = xc' + c, \qquad y_p'' = xc'' + 2c'.$$
Insertion into the original DE will then gives an equation for $c'(x)$:
$$(x^2+1)xc'' +2(x^2+1)c' - 2x^2c' - 2xc + 2xc = (x^2+1)^2.$$
So we have the equation
$$c'' + \frac{2}{x(x^2 + 1)}c' = \frac{x^2+1}{x}.$$
This is a first-order linear equation for $c'(x)$, which we solve by using the technique of integrating factors, we see that 
$$P(x) = \frac{2}{x(x^2+1)} \To I = \int P(x) \ \d x = \int \frac{2}{x(x^2+1)} \ \d x,$$
using the substitution $u = x^2$ and partial fraction decomposition we find
$$I = \int \frac{1}{u(u+1)} \ \d u = \int \frac{1}{u} \ \d u - \frac{1}{1+u} \ \d u = \ln x^2 - \ln (1+x^2),$$
and
$$e^I = \frac{x^2}{1+x^2}.$$
The solution for $c'(x)$ is then
$$c'(x) \frac{x^2}{1+x^2} = \int \frac{x^2+1}{x} \frac{x^2}{1+x^2} \ \d x + D = \frac{1}{2}x^2 + D,$$
$$c'(x) = \frac{1+x^2}{2} + D\frac{1+x^2}{x^2},$$
as we are looking for a particular solution, we can discard the second term, as the integration constant can be chosen ($D=0$), giving us
$$c'(x) = \frac{1+x^2}{2} \To c(x) = \int \frac{1+x^2}{2}\ \d x = \frac{1}{2}x + \frac{1}{6}x^3,$$
meaning we have found a particular solution
$$y_p(x) = xc(x) = \frac{1}{2}x^2 + \frac{1}{6}x^4,$$
and the full solution of the DE is then
$$y(x) = c_1 x + c_2(1-x^2) + \frac{1}{2}x^2 + \frac{1}{6}x^4.$$

\clearpage

\section*{Boas 8.13.7}
We will solve the non-linear first-order ODE:
$$3x^3y^2y' - x^2y^3 = 1.$$
We start by using the substitution
$$u = y^3, \qquad u' = 3xy^2y',$$
giving a linear ODE, on standard form, we have
$$u' - \frac{u}{x} = \frac{1}{x^2}.$$
We can now solve this equation for $u(x)$ using integrating factors
$$I(x) = \int P(x) \ \d x = - \int \frac{1}{x} \ \d x = -\ln x,$$
and
$$e^{I(x)} = \frac{1}{x}.$$
So the solution is given by
$$u(x) = x\bigg(\int \frac{1}{x^3} \ \d x + c\bigg) = cx - \frac{1}{3x^2},$$
substituting back for $y$, gives
$$y(x) = c\sqrt[3]{x} - \frac{1}{\sqrt[3]{3x^2}}.$$

\clearpage

\section*{Boas 12.1.8}
We will solve the DE:
$$(x+2x)y'' - 2(x+1)y' + 2y = 0,$$
using the power series method.

We start by writing the solution as a general power series
$$y(x) = \sum_{n=0}^\infty a_n x^n,$$
and insert this into the original DE, if we then compare power by power, we get the equation
$$\bigg(n^2 - 3n + 2\bigg)a_n + \bigg(2n^2 -2\bigg)a_{n+1} = 0,$$
We can then insert for $n=0,1,2,\ldots$

\begin{center}
\begin{tabular}{l | l | l}
$n = 0:$ & $2a_0 - 2a_1 = 0$ & $a_1 = a_0$\\
$n = 1:$ & $0a_1 - 0a_2 = 0$ & $a_2 = c_2$ \\
$n = 2:$ & $0a_2 + 2a_3 = 0$ & $a_3 = 0$ \\
$n = 3:$ & $2a_3 + 16a_4 = 0$ & $a_4 = 0$ \\ 
\end{tabular}
\end{center}

and we see this trend will continiue forever, meaning all coefficients $a_n$ for $n\geq3$ are equal to zero. Giving us the final solution
$$y(x) = c_1(1+x) + c_2x^2.$$



\clearpage

\section*{Boas 12.11.2}
We will solve the DE:
$$x^2y'' + xy' -9y = 0,$$
using Frobenius' method. We write $y(x)$ as
$$y(x) = x^s \sum_{m=0}^\infty a_m x^m = \sum_{m=0}^\infty a_m x^{m+s}.$$
Inserting this into the original DE gives
$$\sum_{m=0}^\infty [(m+s)(m+s-1)a_m + (m+s)a_m -9a_m]x^{m+s}.$$
We can now compare power by power, starting with $m=0$:
$$s(s+1)a_0 + sa_0 - 9a_0 = 0,$$
as we know $a_0$ is 0 by assumption, we now have an indicial equation that we can use to find $s$:
$$s^2 - 9 = (x+3)(x-3) = 0 \To s = \pm 3.$$
We use the smallest value of $s=-3$, and derive an expression for $a_n$, by comparing the powers of $x^{n-3}$:
$$(n-3)(n-4)a_n + (n-3)a_n -9 a_n = 0,$$
which can be simplified to
$$n(n-6)a_n = 0,$$
so we see that $a_n = 0$ for every term except for $n=0$ and $n=6$, so we have $a_0$ and $a_6$ as free coefficients. We now get
$$y(x) = x^{-3}\bigg(a_0 + a_6 x^6\bigg) = c_1 x^3 + \frac{c_2}{x^3}.$$



\end{document}

