\documentclass[a4paper, 11pt, titlepage, english]{article}

\usepackage{babel}
\usepackage[latin1]{inputenc}
\usepackage[T1]{fontenc, url}
\usepackage{textcomp}
\usepackage{amsmath, amssymb}
\usepackage{amsbsy, amsfonts}
\usepackage{graphicx, color}
\usepackage{parskip}
\usepackage{framed}
\usepackage{amsmath}
\usepackage{xcolor}
\usepackage{multicol}
\usepackage{url}
\usepackage{flafter}


\usepackage{geometry}
\geometry{headheight=0.01mm}
\geometry{top=24mm, bottom=30mm, left=39mm, right=39mm}






%
% Parametere for inkludering av kode fra fil
%
\usepackage{listings}
\lstset{language=python}
\lstset{basicstyle=\ttfamily\small}
\lstset{frame=single}
\lstset{keywordstyle=\color{red}\bfseries}
\lstset{commentstyle=\itshape\color{blue}}
\lstset{showspaces=false}
\lstset{showstringspaces=false}
\lstset{showtabs=false}
\lstset{breaklines}

%
% Definering av egne kommandoer og miljøer
%
\newcommand{\dd}[1]{\ \text{d}#1}
\newcommand{\f}[2]{\frac{#1}{#2}} 
\newcommand{\beq}{\begin{equation*}}
\newcommand{\eeq}{\end{equation*}}
\newcommand{\bra}[1]{\langle #1|}
\newcommand{\ket}[1]{|#1 \rangle}
\newcommand{\braket}[2]{\langle #1 | #2 \rangle}
\newcommand{\braup}[1]{\langle #1 \left|\uparrow\rangle\right.}
\newcommand{\bradown}[1]{\langle #1 \left|\downarrow\rangle\right.}
\newcommand{\av}[1]{\left| #1 \right|}
\newcommand{\op}[1]{\hat{#1}}
\newcommand{\braopket}[3]{\langle #1 | {#2} | #3 \rangle}
\newcommand{\ketbra}[2]{\ket{#1}\bra{#2}}
\newcommand{\pp}[1]{\frac{\partial}{\partial #1}}
\newcommand{\ppn}[1]{\frac{\partial^2}{\partial #1^2}}
\newcommand{\up}{\left|\uparrow\rangle\right.}
\newcommand{\upup}{\left|\uparrow\uparrow\rangle\right.}
\newcommand{\down}{\left|\downarrow\rangle\right.}
\newcommand{\downdown}{\left|\downarrow\downarrow\rangle\right.}
\newcommand{\updown}{\left|\uparrow\downarrow\rangle\right.}
\newcommand{\downup}{\left|\downarrow\uparrow\rangle\right.}
\newcommand{\bupup}{\left.\langle\uparrow\uparrow\right|}
\newcommand{\bdowndown}{\left.\langle\downarrow\downarrow\right|}
\newcommand{\bupdown}{\left.\langle\uparrow\downarrow\right|}
\newcommand{\bdownup}{\left.\langle\downarrow\uparrow\right|}


\makeatletter
\renewcommand*\env@matrix[1][*\c@MaxMatrixCols c]{%
  \hskip -\arraycolsep
  \let\@ifnextchar\new@ifnextchar
  \array{#1}}
\makeatother

%
% Navn og tittel
%
\author{Jonas van den Brink}
\title{Problem set 1 \\ FYS3140}


\begin{document}
\maketitle
\newpage\null\thispagestyle{empty}\newpage

\setcounter{page}{1} 

\section*{Problem 1.1 (Complex power series)}
\subsection*{a)}
We will find the disk of convergence for the power series
$$\sum_{n=0}^\infty n(n+1)(z-2i)^n.$$
We use the ratio test to find for what values of $z$ the series is absolutely convergent
$$\rho = \lim_{n\rightarrow \infty} \bigg|\frac{a_{n+1}}{a_n} \bigg| = \lim_{n\rightarrow \infty} \bigg|\frac{(n+1)(n+2)}{n(n+1)}\bigg|\bigg|\frac{(z-2i)^{n+1}}{(z-2i)^n}\bigg| = \lim_{n\rightarrow \infty} \frac{n+2}{n} |(z-2i)| = |z-2i|$$
The series converges if $\rho<1$, giving the condition on $z$:
$$|z-2i| < 1,$$
meaning the disk of convergence for the power series has a radius of 1 and is centered around $z_0 = 2i$.

\subsection*{b)}
We will find the disk of convergence for the power series
$$\sum_{n=1}^\infty 2^n(z+i-3)^{2n}.$$
Again we use the ratio test
$$\rho = \lim_{n\rightarrow \infty} \bigg|\frac{a_{n+1}}{a_n} \bigg| = \lim_{n\rightarrow \infty} \bigg| \frac{2^{n+1}}{2^n}\bigg|\bigg|\frac{(z+i-3)^{2n+2}}{(z+i-3)^{2n}} \bigg| = \lim_{n\rightarrow \infty} 2|(z+i-3)|^2$$
The series converges if $\rho<1$, giving the condition on $z$:
$$2|z+i-3|^2 < 1 \qquad \Rightarrow \qquad |z-(3-i)| < \frac{\sqrt{2}}{2}$$
meaning the disk of convergence for the power series has a radius of $\sqrt{2}/2$ and is centered around $z_0 = 3-i$.

\clearpage

\section*{Problem 1.2 (Euler's formula, powers and roots)}
\subsection*{a)}
\begin{align*}
z = \sqrt{2}\ {\rm exp}\bigg[\frac{5\pi}{4}i\bigg] = \sqrt{2}\bigg[\cos\bigg(\frac{5\pi}{4}\bigg) + i\sin\bigg(\frac{5\pi}{4}\bigg)\bigg] = -1 - i.    
\end{align*}

\subsection*{b)}
We want to simplify the fraction
$$z = \frac{(1+i)^{48}}{(\sqrt{3}-i)^{25}}.$$
We start by converting the base numbers to polar form
$$1+i = \sqrt{2}e^{i \pi/4}, \qquad \qquad \sqrt{3}-i = 2e^{i11\pi/6}.$$
We can now easily calculate
$$(1+i)^{48} = \big(\sqrt{2}\big)^{48} e^{i 12\pi} = 2^{24}$$
and
$$(\sqrt{3}-i)^{25} = \big(2e^{i 11\pi/6}\big)^{25} = 2^{25}e^{275\pi/6} = 2^{25}e^{-i\pi/6}.$$
Meaning the fraction can be written
$$z = \frac{2^{24}}{2^{25}e^{-i\pi/6}} = \frac{1}{2}e^{i\pi/6} = \frac{\sqrt{3}}{4} + \frac{i}{4}.$$

\subsection*{c)}
We start by writing the base in polar form
$$r = \sqrt{8^2\cdot 3 + 8^2} = 16, \qquad \theta = \frac{2\pi}{3}.$$
We can now easily find the roots
$$\bigg(8i-\sqrt{3}-8)^{1/4} = \bigg(16e^{i2\pi/3}\bigg)^{1/4} = 2 e^{i\pi/6 + k\cdot i\frac{\pi}{2}}$$
where $k=0,1,2,3$, so we have the roots
$$z_0 = 2e^{i\pi/6}, \qquad z_1 = 2e^{i2\pi/3}, \qquad z_2 = 2e^{i7\pi/6}, \qquad z_3 = 2e^{i 5\pi/3}.$$ 

\clearpage
\subsection*{d)}
The cube roots of 8 are
$$8^{1/3} = \bigg(8e^{i 0/\pi}\bigg)^{1/3} = 2e^{k\cdot i2\pi/3},$$
for $k=0,1,2$, meaning the roots are
$$z_0 = 2, \qquad z_1 = 2e^{2\pi/3}, \qquad z_2 = 2e^{4\pi/3}.$$
The sum of the three roots becomes
$$z_0 + z_1 + z_2 = 2 + 2\left(-\frac{1}{2} + \frac{\sqrt{3}}{2}i\right) + 2\left(-\frac{1}{2} - \frac{\sqrt{3}}{2}i\right) = 0,$$ 
as expected.

We will now show the general case that the sum of the $n$ $n$th roots of any complex number will be zero. A geneal complex number,  $z \in \mathbb{C}$, can be written in polar form as
$$z = re^{i\theta},$$
where $r, \theta \in \mathbb{R}$. The $n$th roots of $z$ can then be written
$$z^{1/n} = r^{1/n} e^{i\theta/n + ik\cdot2\pi/n},$$
where the $n$ different roots correspond to $k=0,1,\ldots,n-1$. This means that the sum of the $n$ roots can be written
$$\sum_{k=0}^{n-1} r^{1/n}\ {\rm exp}\bigg[\frac{i\theta}{n} + i\frac{k\cdot2\pi}{n}\bigg] = r^{1/n}\ {\rm exp}\bigg[\frac{i\theta}{n}\bigg] \sum_{k=0}^{n-1}  {\rm exp}\bigg[i\frac{k\cdot2\pi}{n}\bigg].$$
The factor 
$$ {\rm exp}\bigg[\frac{i\theta}{n}\bigg],$$
is never zero, and the factor
$$ r^{1/n}$$
is zero if, and only if, $z = 0$. 

What we need to show then, is that the sum 
$$\sum_{k=0}^{n-1}  {\rm exp}\bigg[i\frac{k\cdot2\pi}{n}\bigg],$$
is zero for all $z \neq 0$. As the sum has a finite number of finite terms, we know that it must converge to some complex number $w \in \mathbb{C}$. As the terms in the sum correspond to the $n$th roots of unity, they are distributed along the unit circle in such a way that rotating all the points an angle $2\pi/n$ won't change the sum. Meaning
$$w\cdot e^{i2\pi/n} = w.$$
The only complex number that remains invariant under a rotation of an angle $\theta \neq 2\pi n$ where $n$ is an integer, is zero. Meaning $w = 0$ for all $n$ and $z$---making the sum of the $n$ $n$th roots of any complex number $z$ equal to zero.

\section*{Problem 1.3 (Elementary Functions)}
We will now use the definitions
$$\sin(z) \equiv \frac{e^{iz}-e^{-iz}}{2i}, \qquad \cos(z)\equiv\frac{e^{iz} + e^{-iz}}{2}, $$
$$ \sinh(z)\equiv\frac{e^z-e^{-z}}{2}, \qquad \cosh(z)\equiv\frac{e^z + e^{-z}}{2}.$$

\subsection*{a)}
We will calculate the integral
$$I = \int_0^{2\pi} \sin^2(4x) {\rm\ d}x.$$ 
We start by rewriting the sine-term in the following manner
$$\sin^2(4x) = \bigg[\frac{e^{i4x}-e^{-i4x}}{2i}\bigg]^2 = \frac{1}{4}\bigg(2 - e^{i8x} - e^{-i8x}\bigg).$$
Using this we find
\begin{align*}
I &=  \int_0^{2\pi} \sin^2(4x) {\rm\ d}x = \frac{1}{4}\int_0^{2\pi} \bigg(2 - e^{i8x} - e^{-i8x}\bigg) {\rm\ d}x, \\[0.2cm]
&= \frac{1}{4}\bigg[2x - \frac{1}{8i}e^{i8x} + \frac{1}{8i}e^{-i8x}\bigg]_{0}^{2\pi} \\[0.3cm]
&= \pi \qquad {\rm q.e.d.}
\end{align*}

\subsection*{b)}
We will now show that the trigonometric identity 
$$\sin 2z = 2\sin z \cos z.$$
holds for all complex numbers, $z\in\mathbb{C}$.

The left-hand side can be written as
$$\sin 2z = \frac{e^{i2z}-e^{-i2z}}{2i},$$
and the right-hand side can be written
$$2\frac{e^{iz} - e^{-iz}}{2i}\frac{e^{iz} + e^{-iz}}{2} = \frac{e^{i2z}-e^{-i2z}}{2i}.$$
So we see that the left-hand side and the right-hand side are the same, and the identity holds, for all $z\in\mathbb{C}$.

\clearpage

\subsection*{c)}
We will now show that the trigonometric identity
$$\cosh^2 z - \sinh^2 z = 1,$$
holds for all $z \in \mathbb{C}$.

Expressing $\sinh$ and $\cosh$ in terms of the exponential functions the left hand side can be written
\begin{align*}
\cosh^2 z - \sinh^2 z &= \bigg(\frac{e^z + e^{-z}}{2}\bigg)^2 - \bigg( \frac{e^z-e^{-z}}{2} \bigg)^2 \\[0.2cm]
&=  \frac{1}{4}\bigg(e^{2z} + 2 + e^{-2z}\bigg) - \frac{1}{4}\bigg(e^{2z} - 2 + e^{-2z}\bigg) \\[0.2cm]
&= 1.
\end{align*}
So we see that the left-hand side and the right-hand side are the same, and the identity holds, for all $z\in\mathbb{C}$.

\subsection*{d)}
We will find the $x+iy$ form of 
$$\sin\bigg( i \ln \frac{1-i}{1+i} \bigg).$$

We start by using the general fact that
$$\sin(iz) = i\sinh(z),$$
giving
$$\sin\bigg( i \ln \frac{1-i}{1+i} \bigg) = i\sinh\bigg(\ln \frac{1-i}{1+i} \bigg).$$
And writing $\sinh$ in terms of exponential functions now gives
$$i\sinh\bigg(\ln \frac{1-i}{1+i} \bigg) = i \frac{e^{\ln (1-i)/(1+i)} - e^{\ln (1+i)/(1-i)}}{2},$$
and using that for any $z \in \mathbb{C}$
$$e^{\ln z} = z,$$
we get
$$\frac{i}{2}\bigg(\frac{1-i}{1+i} - \frac{1+i}{1-i}\bigg) = \frac{i}{4}\bigg((1-i)^2 - (1+i)^2\bigg) = 1.$$

So in summary
$$\sin\bigg( i \ln \frac{1-i}{1+i} \bigg) = 1.$$

\subsection*{e)}
We will find the $x+iy$ form of
$$(-e)^{i\pi}.$$
We start by writing $(-e)$ as $(-1)\cdot e$ and then writing $-1$ in polar form $-1 = e^{i\pi}$, giving
$$(-e)^{i\pi} = (e^{i\pi})^{i\pi}e^{i\pi}.$$
This can be further simplified
$$e^{i^2\pi^2}(-1) = -e^{-\pi^2}.$$
And we are done, as this is a purely real number with numerical value
$$-e^{-\pi^2} \approx -5.17\times10^{-5}.$$

\subsection*{f)}
We now want to express the inverse hyperbolic function ${\rm arctanh}(z)$ in terms of logarithms. Let us introduce the number $w\in\mathbb{C}$ such that
$${\rm arctanh}(z) = w,$$
then by definition
$$\tanh(w) = z.$$
Expressing $\tanh$ in terms of exponentials then gives
$$\tanh(w) = \frac{\sinh(w)}{\cosh(w)} = \frac{e^w - e^{-w}}{2}\frac{2}{e^{w}+e^{-w}} = \frac{e^w - e^{-w}}{e^{w}+e^{-w}}=z.$$
We now have an equation we can solve to express $w$ in terms of $z$ using logarithms
\begin{align*}
\frac{e^w - e^{-w}}{e^{w}+e^{-w}} &= z \\
e^w - e^{-w} &= z\bigg(e^{w}+e^{-w}\bigg) \\
e^{2w} - 1 &= z\bigg(e^{2w}+1\bigg) \\
(1-z)e^{2w} &= 1 + z \\
e^{2w} &= \frac{1+z}{1-z} \\
2w &= \ln \bigg(\frac{1+z}{1-z}\bigg) \\
w &= \frac{1}{2}\ln \bigg(\frac{1+z}{1-z}\bigg).
\end{align*}
Meaning
$${\rm arctanh\ } z = \frac{1}{2}\ln \bigg(\frac{1+z}{1-z}\bigg),\quad {\rm q.e.d.}$$

\clearpage

\section*{Extra Problem (Boas 2.17.30)}

The Taylor expansion of $e^z$ is
$$e^z = \sum_{k=0}^\infty \frac{z^k}{k!} = 1 + z + \frac{z^2}{2!} + \frac{z^3}{3!} + \ldots,$$
meaning we can expand $e^{x(1+i)}$ as
$$e^{x(1+i)} = 1 + x(1+i) + \frac{x^2(1+i)^2}{2!} + \frac{x^3(1+i)^3}{3!} + \ldots$$
To easily obtain the powers of $(1+i)$, it's smart to write it in polar form
$$(1+i) = \sqrt{2}e^{i\pi/4},$$
meaning the $k$th power of $(1+i)$ is
$$(1+i)^k = (\sqrt{2})^k e^{ik\pi/4}.$$

We now want to find the Taylor expansion of $e^x\cos x$, we start by writing $\cos$ in term of exponentials
$$e^x\cos x = e^x \frac{e^{ix} + e^{-ix}}{2} = \frac{e^{x(1+i)} + e^{x(1-i)}}{2},$$
we can now Taylor expand the two terms seperately to find
$$e^x\cos x = \sum_{k=0}^\infty \frac{x^k\big[ (1+i)^k + (1-i)^k\big]}{2k!}$$

To understand the behaviour of the terms in the square bracket as $k$ increases, we also write $(1-i)$ in polar coordinates
$$1-i = \sqrt{2}e^{-i\pi/4}, \qquad (1-i)^k = (\sqrt{2})^k e^{-ik\pi/4}.$$
So we see that the two terms are always the same in magnitude, but that they rotate in opposite directions. The $(1+i)$ term rotates $\pi/4$ counter-clockwise when $k$ is increased by 1, while the $(1-i)$ term rotates clockwise. This behaviour means they are perfectly antisymmetric and cancel each other out for $k=2$ and every 4th power following, i.e.\ $k=6,10,14,\ldots$.

Writing out the terms we get
$$e^x\cos x = \sum_{k=0}^\infty \frac{x^k\big[ (1+i)^k + (1-i)^k\big]}{2k!} = 1 + x - \frac{x^3}{3} - \frac{x^4}{6} - \frac{x^5}{30} + \ldots.$$

\clearpage

Similarly for $e^x \sin x$ we get
$$e^x \sin x = e^x \frac{e^{ix} - e^{-ix}}{2i} = \frac{e^{x(1+i)} - e^{x(1-i)}}{2i}.$$
Taylor expanding the two terms seperately then gives
$$e^x\sin x = \sum_{k=0}^\infty \frac{x^k\big[ (1+i)^k - (1-i)^k\big]}{2ik!}.$$
As the terms are now subtracted instead of added together, they no longer cancel out when they are pointing in opposit directions, but instead when they point in the same direction. Following the same reasoning we used for the cosine-case, we know this happens for $k=0$, $k=4$, and every fourth term thereafter, i.e.\ $k=8,12,16,\ldots$.

Writing out the first terms gives
$$e^x\sin x = \sum_{k=0}^\infty \frac{x^k\big[ (1+i)^k - (1-i)^k\big]}{2ik!} = x + x^2 + \frac{x^3}{3} - \frac{x^5}{30} - \frac{x^6}{90} + \ldots$$

 






\end{document}

