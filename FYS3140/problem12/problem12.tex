\documentclass[a4paper, 11pt, titlepage, english]{article}

\usepackage{babel}
\usepackage[utf8]{inputenc}
\usepackage[T1]{fontenc, url}
\usepackage{textcomp}
\usepackage{amsmath, amssymb}
\usepackage{amsbsy, amsfonts}
\usepackage{graphicx, color}
\usepackage{parskip}
\usepackage{framed}
\usepackage{amsmath}
\usepackage{xcolor}
\usepackage{multicol}
\usepackage{url}
\usepackage{flafter}


\usepackage{geometry}
\geometry{headheight=0.01mm}
\geometry{top=24mm, bottom=30mm, left=39mm, right=39mm}

%
% Parametere for inkludering av kode fra fil
%
\usepackage{listings}
\lstset{language=python}
\lstset{basicstyle=\ttfamily\small}
\lstset{frame=single}
\lstset{keywordstyle=\color{red}\bfseries}
\lstset{commentstyle=\itshape\color{blue}}
\lstset{showspaces=false}
\lstset{showstringspaces=false}
\lstset{showtabs=false}
\lstset{breaklines}

%
% Definering av egne kommandoer og miljøer
%
\newcommand{\dd}[1]{\ \text{d}#1}
\newcommand{\f}[2]{\frac{#1}{#2}} 
\newcommand{\beq}{\begin{equation*}}
\newcommand{\eeq}{\end{equation*}}
\newcommand{\bra}[1]{\langle #1|}
\newcommand{\ket}[1]{|#1 \rangle}
\newcommand{\braket}[2]{\langle #1 | #2 \rangle}
\newcommand{\braup}[1]{\langle #1 \left|\uparrow\rangle\right.}
\newcommand{\bradown}[1]{\langle #1 \left|\downarrow\rangle\right.}
\newcommand{\av}[1]{\left| #1 \right|}
\newcommand{\op}[1]{\hat{#1}}
\newcommand{\braopket}[3]{\langle #1 | {#2} | #3 \rangle}
\newcommand{\ketbra}[2]{\ket{#1}\bra{#2}}
\newcommand{\pp}[1]{\frac{\partial}{\partial #1}}
\newcommand{\ppn}[1]{\frac{\partial^2}{\partial #1^2}}
\newcommand{\up}{\left|\uparrow\rangle\right.}
\newcommand{\upup}{\left|\uparrow\uparrow\rangle\right.}
\newcommand{\down}{\left|\downarrow\rangle\right.}
\newcommand{\downdown}{\left|\downarrow\downarrow\rangle\right.}
\newcommand{\updown}{\left|\uparrow\downarrow\rangle\right.}
\newcommand{\downup}{\left|\downarrow\uparrow\rangle\right.}
\newcommand{\bupup}{\left.\langle\uparrow\uparrow\right|}
\newcommand{\bdowndown}{\left.\langle\downarrow\downarrow\right|}
\newcommand{\bupdown}{\left.\langle\uparrow\downarrow\right|}
\newcommand{\bdownup}{\left.\langle\downarrow\uparrow\right|}
\renewcommand{\d}{{\rm d}}
\newcommand{\Res}[2]{{\rm Res}(#1;#2)}
\newcommand{\To}{\quad\Rightarrow\quad}
\newcommand{\eps}{\epsilon}
\makeatletter
\renewcommand*\env@matrix[1][*\c@MaxMatrixCols c]{%
  \hskip -\arraycolsep
  \let\@ifnextchar\new@ifnextchar
  \array{#1}}
\makeatother


\newcommand{\bt}[1]{\boldsymbol{#1}}
\newcommand{\mat}[1]{\textsf{\textbf{#1}}}
\newcommand{\I}{\boldsymbol{\mathcal{I}}}
\newcommand{\p}{\partial}
%
% Navn og tittel
%
\author{Jonas van den Brink}
\title{Exercise 24 \\ INF5620}


\begin{document}
\maketitle

\section{Background}

A body moving vertically through a fluid is subject to gravity, drag, and buoyancy. The gravity force is a body force and is simply
$$F_g = -mg,$$
where $m$ is the mass of the body, and $g$ is the acceleration of gravity. The buoyancy is a surface force and will act in the upwards direction, the magnitude of the buoyancy force is proportional to the mass of the fluid displaced by the body:
$$F_b = \rho g V,$$
where $\rho$ is the density of the fluid, $g$ is the acceleration of gravity and $V$ is the volume of the body.

The drag force on the object is modelled using two different models. The accuracy of the models depend on the Reynolds number of the object, which is defined as
$${\rm Re} \equiv \frac{\rho d|v|}{\mu},$$
where $d$ is the diameter of the body in the direction perpendicular to the flow, $v$ is the velocity of the body, and $\mu$ is the dynamic viscosity of the fluid. If ${\rm \Re} < 1$, the drag can be modelled fairly well by the Stokes' drag, which for a spherical body reads
$$F_d^{(S)}  = -3\pi d \mu v.$$
While for large Re (> $10^3$), the drag force is quadratic in velocity:
$$F_d^{(q)} = -\frac{1}{2}C_D \rho A|v|v,$$
where $C_D$ is a dimensionless drag coefficient depending on the body's shape, and $A$ is the cross-sectional area as produced by a cut plane, perpendicular to the motion, through the thickest part of the body.

\subsection{Equation of Motion}
We now assume that the Reynolds number is low, so we use Stokes' drag, and apply Newtons 2. law of motion
\begin{align*}
ma &= F_g + F_d^{(S)} + F_b \\
&= -mg - 3\pi d \mu v + \rho g V.
\end{align*}
This equation now contains two unknowns: the velocity $v$ and acceleration $a$. We express the acceleration as the time-derivative of the velocity:
$$m\dot{v} = -mg -3\pi d \mu v + \rho g V.$$
We divide by $m$ to find
$$\dot{v} = - \frac{3 \pi d \mu}{m} + \frac{\rho g V}{m} - g.$$
We now introduce the shorthands
$$b \equiv \frac{3 \pi d \mu}{m}, \qquad c \equiv \frac{\rho g V}{m} - g,$$
so we get
$$\dot{v} = -b v + c.$$
To solve this equation, we will of course need some initial condition
$$v(0) = v_0.$$

\section{Numerical Scheme}
For Stokes' drag, the initial value problem to be solved was given by
$$\dot{v}(t) = -a v(t) + b, \quad  v(0) = v_0, \quad t \in (0, T],$$
where $a$ and $b$ were constants. We now find an appropriate numerical scheme for solving this problem by discretizing the time dimension uniformly, using $N$ cells.

We now use a Crank-Nicolson finite difference approximation for the time derivative of the velocity. This means we sample the ode between the mesh points $t_n$ and $t_{n+1/2}$. We have
$$[D_n v = -a \bar{v}_t + b]^{n+1/2},$$
where $\bar{v}_t$ denotes the time-averaged value of $v$. Using these operations, we find
$$\frac{v^{n+1} - v^n}{\Delta t} = -a \frac{1}{2}(v^{n+1} - v^{n}) + b.$$
Here, $v^{n}$ is generally known, and $v^{n+1}$ is to be found, solving the equation for $v^{n+1}$ gives the expression
$$v^{n+1} = \frac{\big(1 - a\Delta t/2\big)v^n + b\Delta t}{1 + a\Delta t / 2}.$$

For the quadratic drag force, the initial value problem to be solved was given by
$$\dot{v}(t) = -c |v(t)|v(t) + b, \quad  v(0) = v_0, \quad t \in (0, T].$$
The approach to finding a numerical scheme is very similar, but in this case, we use the geometric average for $|v|v$, instead of the arithmetic average, so we get
$$\frac{v^{n+1} - v^n}{\Delta t} = -c|v^n|v^{n+1} + b.$$
Solving for $v^{n+1}$ gives
$$v^{n+1} = \frac{v^n + b\Delta t}{1 + c\Delta t|v^n|}.$$




\end{document}

