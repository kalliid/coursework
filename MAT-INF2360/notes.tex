\documentclass[a4paper, 11pt, notitlepage, english]{article}

\usepackage{babel}
\usepackage[utf8]{inputenc}
\usepackage[T1]{fontenc, url}
\usepackage{textcomp}
\usepackage{amsmath, amssymb}
\usepackage{amsbsy, amsfonts}
\usepackage{graphicx, color}
\usepackage{parskip}
\usepackage{framed}
\usepackage{amsmath}
\usepackage{xcolor}
\usepackage{multicol}
\usepackage{url}
\usepackage{flafter}


\usepackage{geometry}
\geometry{headheight=0.01mm}
\geometry{top=24mm, bottom=29mm, left=39mm, right=39mm}

\renewcommand{\arraystretch}{2}
\setlength{\tabcolsep}{10pt}
\makeatletter
\renewcommand*\env@matrix[1][*\c@MaxMatrixCols c]{%
  \hskip -\arraycolsep
  \let\@ifnextchar\new@ifnextchar
  \array{#1}}
%
% Parametere for inkludering av kode fra fil
%
\usepackage{listings}
\lstset{language=python}
\lstset{basicstyle=\ttfamily\small}
\lstset{frame=single}
\lstset{keywordstyle=\color{red}\bfseries}
\lstset{commentstyle=\itshape\color{blue}}
\lstset{showspaces=false}
\lstset{showstringspaces=false}
\lstset{showtabs=false}
\lstset{breaklines}

%
% Definering av egne kommandoer og miljøer
%
\newcommand{\dd}[1]{\ \text{d}#1}
\newcommand{\f}[2]{\frac{#1}{#2}} 
\newcommand{\beq}{\begin{equation*}}
\newcommand{\eeq}{\end{equation*}}
\newcommand{\bra}[1]{\langle #1|}
\newcommand{\ket}[1]{|#1 \rangle}
\newcommand{\braket}[2]{\langle #1 | #2 \rangle}
\newcommand{\braup}[1]{\langle #1 \left|\uparrow\rangle\right.}
\newcommand{\bradown}[1]{\langle #1 \left|\downarrow\rangle\right.}
\newcommand{\av}[1]{\left| #1 \right|}
\newcommand{\op}[1]{\hat{#1}}
\newcommand{\braopket}[3]{\langle #1 | {#2} | #3 \rangle}
\newcommand{\ketbra}[2]{\ket{#1}\bra{#2}}
\newcommand{\pp}[1]{\frac{\partial}{\partial #1}}
\newcommand{\ppn}[1]{\frac{\partial^2}{\partial #1^2}}
\newcommand{\up}{\left|\uparrow\rangle\right.}
\newcommand{\upup}{\left|\uparrow\uparrow\rangle\right.}
\newcommand{\down}{\left|\downarrow\rangle\right.}
\newcommand{\downdown}{\left|\downarrow\downarrow\rangle\right.}
\newcommand{\updown}{\left|\uparrow\downarrow\rangle\right.}
\newcommand{\downup}{\left|\downarrow\uparrow\rangle\right.}
\newcommand{\bupup}{\left.\langle\uparrow\uparrow\right|}
\newcommand{\bdowndown}{\left.\langle\downarrow\downarrow\right|}
\newcommand{\bupdown}{\left.\langle\uparrow\downarrow\right|}
\newcommand{\bdownup}{\left.\langle\downarrow\uparrow\right|}
\renewcommand{\d}{{\rm d}}
\newcommand{\Res}[2]{{\rm Res}(#1;#2)}
\newcommand{\To}{\quad\Rightarrow\quad}
\newcommand{\eps}{\epsilon}
\newcommand{\inner}[2]{\langle #1 , #2 \rangle}


\newcommand{\bt}[1]{\boldsymbol{#1}}
\newcommand{\mat}[1]{\textsf{\textbf{#1}}}
\newcommand{\I}{\boldsymbol{\mathcal{I}}}
\newcommand{\p}{\partial}
%
% Navn og tittel
%
\author{}
\title{Notes in MAT-INF3360}


\begin{document}
The set of continuous, real functions defined on an interval $[0,T]$ is denoted $C[0,T]$. A real function $f$ defined on $[0,T]$ is said to be \emph{square integrable} if $f^2$ is Riemann-integrable, i.e., if
$$\int_0^T f(t)^2 \ \d t < \infty.$$
The set of all square integrable functions on $[0,T]$ is denoted $L^2[0,T]$.

Both $L^2[0,T]$ and $C[0,T]$ are vector spaces, and we define the inner product on the spaces as
$$\inner{f}{g} = \frac{1}{T}\int_0^T f(t)g(t) \ \d t,$$
and the associated norm
$$||f|| = \sqrt{\frac{1}{T} \int_0^T f(t)^2 \ \d t}.$$
The reason for the $1/T$ normalization factor, is that it makes the constant-function $f(t)=1$ have the norm 1.

The projection of a function $f$ onto a subspace $W$ is the function $g\in W$ which minimizes the least squares error $||f-g||$. It follows that the error function is orthogonal to the subspace $W$,
$$\inner{f-g}{h} = 0, \quad \forall \ h \in W.$$
If $\{\phi_i\}_{i=1}^m$ is an orthogonal basis for $W$, then
$$g = \sum_{i=1}^m \frac{\inner{f}{\phi_i}}{\inner{\phi_i}{\phi_i}}\phi_i.$$

\section*{Fourier}
The N'th order Fourier space is denoted $V_{N,T}$, it is $2N+1$ dimensional and spanned by the set of functions
\begin{align*}
\mathcal{D}_{N,T} = \{1, &\cos\bigg(\frac{2\pi t}{T}\bigg), \cos\bigg(\frac{2\pi 2 t}{T}\bigg)\ldots, \cos\bigg(\frac{2\pi Nt}{T}\bigg), \\
&\sin\bigg(\frac{2\pi t}{T}\bigg), \sin\bigg(\frac{2\pi 2t}{T}\bigg), \ldots \sin\bigg(\frac{2\pi Nt}{T}\bigg)\}.
\end{align*}
it is allso spanned by the complex Fourier basis
$$\mathcal{F}_{N,T} = \bigg\{e^{-2\pi i k t/T}\bigg\}_{k=-N}^N,$$

The projection of a function $f$ onto $V_{N,T}$ is denoted $f_N(t)$, in the real basis we have:
$$f_N(t) = a_0 + \sum_{n=1}^N \bigg[ a_n \cos\bigg(\frac{2\pi n t}{T}\bigg) + b_n \sin\bigg(\frac{2\pi n t}{T}\bigg)\bigg].$$
The real Fourier coefficients of $f$ are given by
\begin{align*}
a_0 &= \inner{f}{1}, \\
a_n &= 2\inner{f}{\cos(2\pi n t/T)}, \\
b_n &= 2\inner{f}{\cos(2\pi n t/T)}.
\end{align*}
In the complex basis we have
$$f_N(t) = \sum_{-N}^N y_n e^{2\pi i n t /T},$$
where the complex Fourier coefficients of $f$ are given by
$$y_n = \langle f, e^{2\pi i n t/T} \rangle= \frac{1}{T}\int_0^T f(t) e^{-2\pi i n t/T}\, \d t. $$
We can map between real and complex Fourier coefficients from
$$
\begin{pmatrix}
y_n \\ y_{-n}   
\end{pmatrix}
 = \frac{1}{2}\begin{pmatrix}
   1 & -i \\ 1 & i  
 \end{pmatrix}
 \begin{pmatrix}
     a_n \\ b_n
 \end{pmatrix}.$$
and $y_0 = a_0$.

\subsubsection*{Convergence of Fourier series}
Given a periodic function $f$ with period $T$, and that 
\begin{itemize}
 \item $f$ has a finite set of discontinuities in each period.
 \item $f$ contains a finite set of maxima and minima in each period.
 \item $\int_0^T |f(t)| \ \d t < \infty$
\end{itemize}
Then we have that $\lim_{N\to\infty} f_N(t) = f(t)$ for all $t$, except at those points $t$ where $f$ is discontinuous. These are the Dirichlet conditions for the convergence of the Fourier series.

If $f$ is antisymmetric about 0, then $a_n = 0$, i.e., the Fourier series becomes a sine-series, if $f$ is symmetric about 0, then $b_n = 0$ and the Fourier series becomes a cosine-series.

\subsection*{Pure tones}
The function
$$e^{2\pi i n t/T},$$
is called a pure tone with frequency n/T.

For complex vectors of length $N$, the Euclidean inner product is
$$\langle \bt{x}, \bt{y} \rangle = \sum_{k=0}^{N-1} x_k \overline{y_k}.$$
And so the associated norm is
$$||\bt{x}|| = \sqrt{\sum_{k=0}^{N-1} |x_k|^2}.$$

\subsection*{Pure digital tones of order $N$}
The pure digital tones of order $N$, also called the normalised complex exponentials, are:
$$\bt{\phi}_n = \frac{1}{\sqrt{N}}\big(1, e^{2\pi i n/N}, e^{2\pi i 2n/N}), \ldots, e^{2\pi i n(N-1)/N}.)$$
The whole collection
$$\mathcal{F}_N = \big\{\bt{\phi}_n\big\}_{n=0}^{N-1},$$
is called the $N$-point Fourier basis. The basis is orthonormal in $\mathbb{R}^N.$

\section*{Discrete Fourier Transform}
The change of coordinates from the standard basis of $\mathbb{R}^N$ to the Fourier basis $\mathcal{F}_N$ is called the discrete Fourier transform, or DFT. The $N\times N$ matrix $F_N$ that represents this change of basis is called the $N$-point Fourier matrix. If $\bt{x} \in \mathbb{R}^n$, then the DFT coefficients of $\bt{x}$ is given as:
$$\bt{y} = F_N \bt{x}$$
We see that the coloumns of the inverse matrix are the pure digital tones
$$\bt{x} = \sum_{k=0}^{N-1}y_k\bt{\phi}_k = \begin{bmatrix}
    \bt{\phi}_0 & \bt{\phi}_1 & \ldots & \bt{\phi}_{N-1}
\end{bmatrix}\bt{y} = F_N^{-1} \bt{y}.$$
As $F_N$ is orthogonal and complex (i.e.\ unitary), we find it's inverse by taking the conjugate transpose of it.
$$F_N = (F_n^{-1})^H.$$
The entires of the $N\times N$ Fourier matrix $F_N$ is given by
$$(F_N)_{nk} = \frac{1}{\sqrt{N}}e^{-2\pi i n k/N}.$$

\subsection*{Properties of DFT}
\begin{itemize}
    \item $\bt{\hat{x}}_{N-n} = \overline{\bt{\hat{x}}_{N-n}}$
\end{itemize}
\end{document}

