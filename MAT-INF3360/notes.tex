\documentclass[a4paper, 11pt, notitlepage, english]{article}

\usepackage{babel}
\usepackage[utf8]{inputenc}
\usepackage[T1]{fontenc, url}
\usepackage{textcomp}
\usepackage{amsmath, amssymb}
\usepackage{amsbsy, amsfonts}
\usepackage{graphicx, color}
\usepackage{parskip}
\usepackage{framed}
\usepackage{amsmath}
\usepackage{xcolor}
\usepackage{multicol}
\usepackage{url}
\usepackage{flafter}


\usepackage{geometry}
\geometry{headheight=0.01mm}
\geometry{top=24mm, bottom=29mm, left=39mm, right=39mm}

\renewcommand{\arraystretch}{2}
\setlength{\tabcolsep}{10pt}
\makeatletter
\renewcommand*\env@matrix[1][*\c@MaxMatrixCols c]{%
  \hskip -\arraycolsep
  \let\@ifnextchar\new@ifnextchar
  \array{#1}}
%
% Parametere for inkludering av kode fra fil
%
\usepackage{listings}
\lstset{language=python}
\lstset{basicstyle=\ttfamily\small}
\lstset{frame=single}
\lstset{keywordstyle=\color{red}\bfseries}
\lstset{commentstyle=\itshape\color{blue}}
\lstset{showspaces=false}
\lstset{showstringspaces=false}
\lstset{showtabs=false}
\lstset{breaklines}

%
% Definering av egne kommandoer og miljøer
%
\newcommand{\dd}[1]{\ \text{d}#1}
\newcommand{\f}[2]{\frac{#1}{#2}} 
\newcommand{\beq}{\begin{equation}}
\newcommand{\eeq}{\end{equation}}
\newcommand{\bra}[1]{\langle #1|}
\newcommand{\ket}[1]{|#1 \rangle}
\newcommand{\braket}[2]{\langle #1 | #2 \rangle}
\newcommand{\braup}[1]{\langle #1 \left|\uparrow\rangle\right.}
\newcommand{\bradown}[1]{\langle #1 \left|\downarrow\rangle\right.}
\newcommand{\av}[1]{\left| #1 \right|}
\newcommand{\op}[1]{\hat{#1}}
\newcommand{\braopket}[3]{\langle #1 | {#2} | #3 \rangle}
\newcommand{\ketbra}[2]{\ket{#1}\bra{#2}}
\newcommand{\pp}[1]{\frac{\partial}{\partial #1}}
\newcommand{\ppn}[1]{\frac{\partial^2}{\partial #1^2}}
\newcommand{\up}{\left|\uparrow\rangle\right.}
\newcommand{\upup}{\left|\uparrow\uparrow\rangle\right.}
\newcommand{\down}{\left|\downarrow\rangle\right.}
\newcommand{\downdown}{\left|\downarrow\downarrow\rangle\right.}
\newcommand{\updown}{\left|\uparrow\downarrow\rangle\right.}
\newcommand{\downup}{\left|\downarrow\uparrow\rangle\right.}
\newcommand{\bupup}{\left.\langle\uparrow\uparrow\right|}
\newcommand{\bdowndown}{\left.\langle\downarrow\downarrow\right|}
\newcommand{\bupdown}{\left.\langle\uparrow\downarrow\right|}
\newcommand{\bdownup}{\left.\langle\downarrow\uparrow\right|}
\renewcommand{\d}{{\rm d}}
\newcommand{\Res}[2]{{\rm Res}(#1;#2)}
\newcommand{\To}{\quad\Rightarrow\quad}
\newcommand{\eps}{\epsilon}
\newcommand{\inner}[2]{\langle #1 , #2 \rangle}


\newcommand{\bt}[1]{\boldsymbol{#1}}
\newcommand{\mat}[1]{\textsf{\textbf{#1}}}
\newcommand{\I}{\boldsymbol{\mathcal{I}}}
\newcommand{\p}{\partial}
%
% Navn og tittel
%
\author{}
\title{Notes in MAT-INF3360}


\begin{document}

\section{Poisson's Equation in 1D}
The two-point boundary value problem:
$$-u''(x) = f(x), \qquad x\in(0,1), u(0) = u(1) = 0.$$
has a general solution that admits the form
\beq
u(x) = x\int_0^1 (1-y)f(y) \, \d y - \int_0^x (x-y) f(y) \, \d y.
\label{eq:1Dpoisson_gensol}
\eeq

\subsection{Proof}
Generally we have that (Fundamental theorem of calculus):
$$u(x) = c_1 + \int_0^x u'(y) \, \d y, \qquad u'(x) = c_2 + \int_0^x u''(z) \, \d z,$$
Inserting our ODE, we get
$$u'(x) = c_1 + c_2 x - \int_0^x \int_0^y f(z) \, \d z \, \d y.$$
And we have
\begin{align*}
\int_0^x \int_0^y f(z) \, \d z \, \d y &= \int_0^x F(y) \, \d y \\
&= [yF(y)]^x_0 - \int_0^x yF'(y) \, \d y \\
&= xF(x) - \int_0^x yf(y) \d y \\
&= \int_0^x (x-y)f(y) \, \d y.
\end{align*}
Combining this with our boundary values $u(0)=u(1)=0$ gives a general solution on the form
$$u(x) = x\int_0^1(1-y)\, \d y - \int_0^x (x-y)\, \d y.$$

\subsection{Green's function}
\beq
G(x,y) = \begin{cases}
y(1-x) & \mbox{if } 0\leq y \leq x, \\
x(1-y) & \mbox{if }  x\leq y \leq 1. \\
\end{cases}
\eeq
Using the Green's function, we can rewrite the general solution \ref{eq:1Dpoisson_gensol} as
\beq
u(x) \texttt{}= \int_0^1 G(x,y)f(y) \, \d y. \label{eq:green_sol}
\eeq

\begin{itemize}
\item $G$ is continuous.
\item $G$ is symmetric, $G(x,y)=G(y,x).$
\item $G(0,y)=G(x,0)=G(1,y)=G(x,1) = 0.$
\item $G(x,y)>0$ for all $x,y\in[0,1].$
\end{itemize}

\clearpage

\subsection{Spaces}
\begin{itemize}
    \item $C((0,1))$ denotes the set of continous functions on the open unit interval.
    \item $C([0,1])$ denotes the space of continuous functions on the closed unit interval.
    \item $C^m((0,1))$ denotes the set of $m$-times continuously differentiable functions on the open unit interval.
    \item $C^2_0((0,1)) = \{ g\in C^2((0,1)) \cap C([0,1])|g(0)=g(1)=0\}.$
\end{itemize}

\subsection{Some characteristics of the solution}
\begin{itemize}
\item For every $f\in ([0,1])$, there is a unique solution $u \in C_0^2((0,1))$, and the solution admits the representation
$$u(x) \texttt{}= \int_0^1 G(x,y)f(y) \, \d y.$$
\item If $f\in C^m((0,1))$ for $m\geq 1$, then $u\in C^{m+2}((0,1))$ and
$$u^(m+2) = -f^(m).$$
Hence, the solution is always smoother than the data.
\item If $f$ is a nonnegative function, then the corresponding solution of $u$ is also nonnegative.
\item $$||u||_\infty \leq (1/8)||f||_\infty.$$
\end{itemize}

\section{Linear algebra}
\subsection{Sup norm}
For a continous function $f\in C([0,1])$ we define the norms:
$$||f||_p = \bigg(\int_0^1 |f(x)|^p \, \d x\bigg)^{1/p}.$$
If we let $p \to \infty$, we get the sup norm:
$$||f||_\infty = \sup_{x\in[0,1]} |f(x)|.$$





\end{document}

